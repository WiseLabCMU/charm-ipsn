
\section{Introduction}
\label{sec:intro}

Low Power Wide Area Networks (LPWANs) are increasingly seen as an attractive
communication platform for city-scale Internet-of-Things (IoT) deployments.
They offer the ability to wirelessly connect energy-constrained devices to
gateways over distances of many kilometers. LPWANs also have power and cost
advantages over alternatives like cellular networks, particularly in
deploy-once, low-maintenance and low throughput sensing applications.

While far from pervasive, the capabilities of LPWANs like
LoRaWAN~\cite{Sornin2015, LoRaWanAlliance2015}, SigFox~\cite{centenaro2016}
and Ingenu's RPMA~\cite{Ingenu2015} have attracted investment and have spawned
early deployments. These technologies operate on unlicensed spectrum, allowing
businesses and consumers alike to deploy their own devices and gateways. With
the recent announcement by Comcast \cite{comcast, comcast2} to integrate
LPWAN radios into future set-top boxes in the U.S., LPWANs are likely to
scale rapidly. In fact, major cities in the U.S. are likely to see fast-paced
LPWAN coverage, given that each LPWAN gateway promises a range of up to ten
kilometers~\cite{LoRaWanAlliance2015}.

Despite the expected rapid rise in density of LPWAN gateways, not all
devices will experience the promised ten-year battery life. Specifically,
devices located in urban spaces deep inside buildings or in remote
neighborhoods will experience severe drain in battery as their signals are
highly attenuated even at the closest base station. Some of these devices --
such as those in basements or tunnels -- may not be in communication range of
any gateway at all. Unlike cellular networks, LPWANs are largely user
deployed and unplanned, meaning that these devices may remain battery-deprived
or simply out of network reach in perpetuity, even as thousands of gateways
proliferate city-wide.

This paper presents Charm, a system that enhances the coverage of LPWANs and
the battery life of clients in large urban deployments. Charm exploits the
observation that while signals from certain clients may attenuate
significantly, they are still likely to be received by multiple gateways in a
dense network. Charm develops a hardware and software design at the gateways
that identifies and transports weak received signals to the cloud. We then
develop a joint decoding system at the cloud that coherently combines weak
signals received across multiple city gateways to decode the underlying data.
As a result, Charm both expands the decoding range of the LPWAN network and
improves battery-life for nodes already in range -- allowing clients across
the network to spend less energy per transmitted bit. We implement Charm on
the LoRaWAN platform~\cite{LoRaWanAlliance2015}, a popular and widely
available LPWAN technology. Charm is implemented in a first-of-its-kind pilot
deployment for coherent diversity combining. It demonstrates increased
network coverage and improved data-rates across the clients.

\begin{figure}
    \centering
    \includegraphics[width=0.60\columnwidth]{figures/LoRaRAN.pdf}
        \vspace*{-0.1in}
    \caption{Charm: LPWAN joint decoding in the cloud}
    \vspace*{-0.1in}
    \label{fig:my_label}
    \compactimg
\end{figure}

While coherent diversity combining and PHY-layer processing in the cloud has
received much attention in the Wi-Fi~\cite{tan2009sam, xie2014scalable} and
cellular~\cite{checko2015cloud, wubben2014benefits} context, designing such a
system for low-power WANs offers radically new challenges. At the gateways, we
would have to decode very weak signals, weaker than 30 dB below the noise
floor. Simply uploading all received data to the cloud would overwhelm
the backend link, which is often  a simple home LAN. Both the LPWAN gateways
and clients are designed to be economical and deployed at scale, and without
the time synchronization required for coherent combining. At the cloud,
collating receptions from a large number of gateways at city-scale to identify
which of them contain packets from the same client is a challenge. We provide
an overview of our approach to address each of these challenges below.

\noindent \textbf{Noise-Resilience at the Gateway:} The key challenge at the
gateway is identifying packets that are significantly below the noise and
therefore virtually undetectable. A strawman approach to this problem would be
to correlate the received signal with a known preamble in any valid packet.
For instance, LoRaWAN uses a sequence of identical chirps -- signals whose
frequency increases linearly in time -- as a signature prefixing every packet.
In principle, sending an extremely long preamble could provide high resilience
to noise. In practice, doing so goes against the spirit of LPWANs where
energy for transmission is a valuable resource for the client.

Charm's approach to resolve this challenge is a hardware and software gateway
design that leverages the structure of the LoRaWAN LPWAN protocol.
Specifically, we develop a transform that converts the data symbols containing
\textit{a priori} unknown bits into a repeated and known sequence of signals
much like the preamble. Charm can therefore now use both the preamble and the
modified data sequence to detect any packet.

% adwait feb 6: seems far too complex for the intro
To understand our approach at a high-level, we present an illustrative example
that dives into the details of the LoRaWAN PHY-layer. Specifically, LoRaWAN
transmits data symbols as chirps whose initial frequency is a function of the
data. For instance over a bandwidth of 100 Hz, LoRa could represent the bit
"0" as a chirp starting at 2 Hz and bit "1" as a chirp starting at 52 Hz.
Charm's filter aliases the received LoRa signal so that frequencies modulo 50
Hz fold into each other. This means that both bit "0" and bit "1" now map to
an identical chirp starting at 2 Hz. We apply this filter through
the received packet to obtain a repeated sequence of chirps as long as the
entire packet itself. This allows us to detect the packet with a much higher
resilience to noise compared to using the preamble alone, without incurring
additional overhead.

We develop a custom gateway hardware platform integrating a Semtech LoRaWAN
radio frontend, a low-power FPGA and Raspberry PI that can filter and detect
weak signals by processing received raw I/Q samples in real-time. Our hardware
platform, a hybrid between a full SDR and a dedicated high-performance radio,
is designed to be open and highly programmable -- a novel tool to experiment
with alternative LPWAN PHY-layer designs in the 900 MHz ISM band, without
compromising on signal quality or real-time performance.

\noindent \textbf{Scalability at the Cloud:} At the cloud, Charm must deal
with a large number of receptions from various gateways in a city, pruning for
weak signals and identifying common signals between gateways. Charm proposes
multiple optimizations to run its algorithms seamlessly at city-scale. For
instance, it is often the case that gateways transmit weak signals to the
cloud for packets that have already been decoded perfectly at other gateways.
However, realizing that the weak signal has already been decoded elsewhere is
impossible without decoding it in the cloud in the first place. Charm resolves
this chicken-or-egg dilemma by exploiting both the timing and geographical
location of the received signal. Prior to sending any signal data to the
cloud, a Charm gateway sends the location, accurate timing and signal-to-noise
ratio of the received weak packet. The cloud collates such information across
multiple gateways and requests for signals only from the gateways that receive
these signals the best. In doing so Charm saves valuable uplink bandwidth at
the gateways and compute cycles at the cloud. We describe how Charm mitigates
range of other important challenges at the cloud such as imperfect timing,
frequency offsets and overlapping transmissions.

We evaluate Charm on the Carnegie Mellon University campus and around the city
of Pittsburgh, in both indoor and outdoor environments using two testbeds.
Eight user-deployed gateways built using our custom hardware platform support
a testbed covering a 0.6 sq.km. area around campus, which is used to study
Charm's performance with regard to local packet detection, range and data-rates. Four
rooftop gateways support the OpenChirp LPWAN network which services a large
10 sq.km. area that we use to acquire traces for large-scale simulations. Our
results reveal the following:

\begin{itemize}
    \item {\bf Battery-Life: }Charm improves the signal to noise ratio at a
        typical LoRaWAN client by coherently combining across 8 base stations
        by 3.16 dB, extending battery life by upto 4$\times$.
    \item {\bf Range: } We improve the maximum communication range of 8 indoor
        user-deployed gateways in urban settings from 60m in LoRaWAN to 200
        meters using Charm, an overall increase in coverage area by
        10$\times$.
    \item {\bf Coverage: } Our trace-driven simulation based on city-wide
        drive tests, estimates an overall increase in coverage by up to 2x due
        to Charm over LoRaWAN.
\end{itemize}

\noindent \textbf{Contributions:} We make the following novel contributions:
\begin{itemize}
    \item A technique that leverages the geographical diversity of unplanned
        user-deployed gateways to enable joint decoding of weak transmissions.
        This improves battery-life for users in the network and increases the
        coverage area.
    \item A hardware platform and the underlying algorithms for detecting weak
        LoRaWAN transmissions locally at the gateway.
    \item A software architecture that builds atop of LoRaWAN to enable
        joint-decoding of signals in a scalable manner.
\end{itemize}