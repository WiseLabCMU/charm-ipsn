\section{Related Work}
\label{sec:related-work}

% {\color{blue}
% [ANYTHONY, SWARUN, BOB, ARTUR, ADWAIT, DIANA, MAX]

% Particular topics to consider
% \begin{itemize}
% \item PerCom workshop paper
% \item Localization (Reverse GPS)
% \item WiFi and Cellular existing work
% \end{itemize}
% }

\noindent \textbf{Low-Power Wide-Area Networks: } Recent years have seen much
interest in Low-Power wide area networks (LP-WANs), including the development
of new hardware and standards. Private enterprises such as
Semtech~\cite{LoRaWanAlliance2015} and SigFox~\cite{sanchez2016state} have
developed LP-WAN chipsets that use extremely narrow bands of unlicensed
spectrum. In contrast, cellular standardization bodies have developed two
standards for LP-WAN communication for cellular base stations to communicate
with low-power IoT devices over licensed spectrum:
LTE-M~\cite{GSMAssociation2016} and NB-IOT~\cite{ratasuk2016nb}. Unlike LoRaWAN
and SigFox, these technologies require devices to periodically wake up to
synchronize with the network -- a burden on battery life.

Several recent measurement studies have been conducted to evaluate the
performance and range of LP-WAN networks~\cite{petric2016measurements,
toldov2016performance} and perform theoretical capacity
analysis~\cite{mikhaylov2016analysis}. Early pilot deployment efforts are also
underway with SigFox deploying their hardware to connect security alarms to
the cloud in Spain~\cite{sanchez2016state}, smart blood refrigerators in the
Democratic Republic of the Congo~\cite{ramachandranmupnp} and smart city
applications \cite{centenaro2015long}. These efforts motivate the challenge of
limited range, performance and battery-drain of LP-WAN clients. A recent
system, Choir~\cite{eletreby2017empowering} has demonstrated improving range
and scalability of LP-WANs through collaborations of weak client radios. In
contrast, this paper seeks to use collaboration between gateways without any
modifications to client behavior whatsoever to improve the battery life of
even a single client.

\noindent \textbf{Distributed MIMO and Coherent Combining: } A large body of
work has proposed the use of multiple-antennas (MIMO) to improve SNR and
reduce interference~\cite{xie2014scalable, lin2011random, kumar2013bringing}.
In the Wi-Fi context, past systems have used multi-user MIMO to improve
performance on the uplink~\cite{shen2014rate, tan2009sam, xie2014scalable}. In
the cellular context, massive MIMO proposals have demonstrated scaling gains
of towers with a large number of antennas~\cite{shepard2012argos,
larsson2014massive}. There has been much theoretical work on distributed MIMO
overall in both the sensor networking context~\cite{del2007cooperative} and
wireless LANs~\cite{dohler2004resource} and cellular
networks~\cite{sawahashi2010coordinated}.  More recently, practical
distributed MIMO systems, primarily in the LAN-context have demonstrated both
multiplexing and diversity gains~\cite{hamed2016real, yenamandra2014vidyut,
rahul2012jmb}. Instead, our approach brings the diversity gains of distributed
MIMO on the uplink to LP-WANs. In doing so, we overcome multiple challenges
owing to the fact that signals at any individual tower are well below the
noise floor and are captured by low-cost hardware that lacks the precise time
synchronization required for coherent combining.

\noindent \textbf{Cloud Radio Access Networks (Cloud-RAN): } Multiple research
efforts from the industry and academia have advocated the use of PHY layer
processing at the cloud as opposed to the base stations~\cite{sabella2013ran,
hadzialic2013cloud}. In the cellular context, CloudRAN aims to perform
baseband processing at the cloud, allowing base stations to be simple and easy
to deploy~\cite{checko2015cloud, wubben2014benefits}. The key challenge
however is the need for a reliable fiber optic backhaul to the cloud to
collate data streams in a low latency manner, motivating the need for
cost-effective high-performance backhauls~\cite{liu2013case, chih2014recent}.
Our approach aims to bring PHY processing in the cloud to LP-WANs that operate
at significantly lower bandwidth, with loose latency bounds and can therefore
afford Ethernet backhauls. We perform a wide variety of optimizations to
minimize the use of uplink bandwidth, including local packet detection and
data compression using an FPGA accelerator. These are helpful when the
gateways are user-deployed with residential internet backbones.