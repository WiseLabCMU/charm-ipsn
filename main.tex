% \documentclass[sigconf, anonymous]{acmart}
\documentclass[sigconf,bookmarks=false]{acmart}

\usepackage{booktabs} % For formal tables
\usepackage{graphicx}
% Note: do not use url & hyperref packages.
% They do not pass the IEEE pdfxpress checks
% \usepackage{url}
% \usepackage{hyperref}
% \usepackage{subfigure}
\usepackage{balance}
\usepackage{amssymb,amsmath}
% \usepackage[usenames,dvipsnames,svgnames,table]{xcolor}
\usepackage{algorithm2e}
\usepackage{algpseudocode}

\usepackage{todonotes} % get rid of this later
\usepackage{subfig}
\usepackage{wrapfig}
\newcommand{\compactimg}{\vspace{-14pt}}
\newcommand{\tableref}[1]{Table~\ref{tab:#1}}
\newcommand{\figref}[1]{Figure~\ref{fig:#1}}
\newcommand{\secref}[1]{Section~\ref{sec:#1}}
\newcommand{\algoref}[1]{Algorithm~\ref{algo:#1}}

\newcommand{\bob}[1]{\textcolor{red}{{\bf from Bob: #1}}}
\newcommand{\cmt}[2]{{\color{blue} #1} {\color{red} [#2]}}

\newenvironment{tightitem}{\begin{itemize}\setlength{\itemsep}{1pt}\setlength{\parskip}{0pt}\setlength{\parsep}{0pt}}{\end{itemize}}

\begin{document}

% Copyright
% \setcopyright{none}
%\setcopyright{acmcopyright}
\setcopyright{acmlicensed}
% \setcopyright{rightsretained}
%\setcopyright{usgov}
%\setcopyright{usgovmixed}
%\setcopyright{cagov}
%\setcopyright{cagovmixed}


% DOI
\acmDOI{None}

% ISBN
\acmISBN{}

%Conference
\acmConference[IPSN 2018]{IEEE/ACM Information Processing in Sensor Networks}{April 2018}{Porto, Portugal}
\acmYear{2018}
\copyrightyear{2018}

% \acmPrice{xx.00}

\title[Charm]{ Charm: Exploiting Geographical Diversity Through \\ Coherent Combining in Low-Power Wide-Area Networks }

%\titlenote{Produces the permission block, and
%  copyright information}
% \subtitle{Extended Abstract}
% \subtitlenote{The full version of the author's guide is available as
%   \texttt{acmart.pdf} document}


\author{Adwait Dongare}
\affiliation{%
  \institution{Carnegie Mellon University}
  %\city{Pittsburgh}
  %\state{PA}
}
\email{adongare@cmu.edu}

\author{Revathy Narayanan}
\affiliation{%
  \institution{Carnegie Mellon University}
  %\city{Pittsburgh}
  %\state{PA}
}
\affiliation{%
  \institution{IIT Madras}
  %\city{Chennai}
  %\state{India}
}
\email{revathyn@andrew.cmu.edu}

\author{Akshay Gadre}
\affiliation{%
  \institution{Carnegie Mellon University}
  %\city{Pittsburgh}
  %\state{PA}
}
\email{agadre@andrew.cmu.edu}

\author{Anh Luong}
\affiliation{%
  \institution{Carnegie Mellon University}
  %\city{Pittsburgh}
  %\state{PA}
}
\email{anhluong@cmu.edu}

\author{Artur Balanuta}
\affiliation{%
  \institution{Carnegie Mellon University}
  %\city{Pittsburgh}
  %\state{PA}
}
\affiliation{%
  \institution{Instituto Superior T\'ecnico Lisboa}
}
\email{artur@cmu.edu}

\author{Swarun Kumar}
\affiliation{%
  \institution{Carnegie Mellon University}
  %\city{Pittsburgh}
  %\state{PA}
}
\email{swarun@cmu.edu}

\author{Bob Iannucci}
\affiliation{%
  \institution{Carnegie Mellon University}
  %\city{Palo Alto}
  %\state{CA}
}
\email{bob.iannucci@west.cmu.edu}

\author{Anthony Rowe}
\affiliation{%
  \institution{Carnegie Mellon University}
  %\city{Pittsburgh}
  %\state{PA}
}
\email{agr@ece.cmu.edu}

% The default list of authors is too long for headers}
\renewcommand{\shortauthors}{Dongare et al.}


\begin{abstract}


Low-Power Wide Area Networks (LPWANs) are an emerging wireless platform which
can support battery-powered devices lasting 10-years while communicating at
low data-rates to gateways several kilometers away. Despite the expected
high density of LPWAN gateways in future cities, not all devices will
experience the promised 10-year battery life, particularly in urban spaces. A
large number of devices deep inside buildings or in remote
neighborhoods will suffer severe battery-drain due to extended transmissions
at the slowest data rate to reach even the closest gateway.

This paper presents Charm, a system that enhances both battery life and
coverage of LPWAN clients in large urban deployments. Charm allows multiple
LoRaWAN gateways to pool their received signals in the cloud, coherently
combining them to detect even the weakest signal that is not decodable at any
individual gateway. Charm achieves this through a novel hardware and software
design at the gateway that carefully detects which chunks of the received
signal need to be sent to the cloud, thereby saving uplink bandwidth. We
demonstrate how our solution scales to decoding weak transmissions at
city-scale, by identifying the set of gateways whose signals need to be
coherently combined over time. In evaluations over a test network and from
simulations using traces from a large LoRaWAN deployment in Pittsburgh, Charm
demonstrates a significant gain of up to 3$\times$ in range and 4$\times$ in client
battery-life.

\end{abstract}

%used to add page numbers
% \settopmatter{printfolios=true}

\begin{CCSXML}
<ccs2012>
<concept>
<concept_id>10010520.10010553.10003238</concept_id>
<concept_desc>Computer systems organization~Sensor networks</concept_desc>
<concept_significance>500</concept_significance>
</concept>
<concept>
<concept_id>10010583.10010588.10010595</concept_id>
<concept_desc>Hardware~Sensor applications and deployments</concept_desc>
<concept_significance>500</concept_significance>
</concept>
<concept>
<concept_id>10003033.10003058.10003065</concept_id>
<concept_desc>Networks~Wireless access points, base stations and infrastructure</concept_desc>
<concept_significance>300</concept_significance>
</concept>
</ccs2012>
\end{CCSXML}

\ccsdesc[500]{Computer systems organization~Sensor networks}
\ccsdesc[500]{Hardware~Sensor applications and deployments}
\ccsdesc[300]{Networks~Wireless access points, base stations and infrastructure}

\keywords{LPWAN, LoRaWAN, sensor networks, coherent combining, diversity}

\maketitle


\section{Introduction}
\label{sec:intro}

Low Power Wide Area Networks (LPWANs) are increasingly seen as an attractive
communication platform for city-scale Internet-of-Things (IoT) deployments.
They offer the ability to wirelessly connect energy-constrained devices to
gateways over distances of many kilometers. LPWANs also have power and cost
advantages over alternatives like cellular networks, particularly in
deploy-once, low-maintenance and low throughput sensing applications.

While far from pervasive, the capabilities of LPWANs like
LoRaWAN~\cite{Sornin2015, LoRaWanAlliance2015}, SigFox~\cite{centenaro2016}
and Ingenu's RPMA~\cite{Ingenu2015} have attracted investment and have spawned
early deployments. These technologies operate on unlicensed spectrum, allowing
businesses and consumers alike to deploy their own devices and gateways. With
the recent announcement by Comcast \cite{comcast, comcast2} to integrate
LPWAN radios into future set-top boxes in the U.S., LPWANs are likely to
scale rapidly. In fact, major cities in the U.S. are likely to see fast-paced
LPWAN coverage, given that each LPWAN gateway promises a range of up to ten
kilometers~\cite{LoRaWanAlliance2015}.

Despite the expected rapid rise in density of LPWAN gateways, not all
devices will experience the promised ten-year battery life. Specifically,
devices located in urban spaces deep inside buildings or in remote
neighborhoods will experience severe drain in battery as their signals are
highly attenuated even at the closest base station. Some of these devices --
such as those in basements or tunnels -- may not be in communication range of
any gateway at all. Unlike cellular networks, LPWANs are largely user
deployed and unplanned, meaning that these devices may remain battery-deprived
or simply out of network reach in perpetuity, even as thousands of gateways
proliferate city-wide.

This paper presents Charm, a system that enhances the coverage of LPWANs and
the battery life of clients in large urban deployments. Charm exploits the
observation that while signals from certain clients may attenuate
significantly, they are still likely to be received by multiple gateways in a
dense network. Charm develops a hardware and software design at the gateways
that identifies and transports weak received signals to the cloud. We then
develop a joint decoding system at the cloud that coherently combines weak
signals received across multiple city gateways to decode the underlying data.
As a result, Charm both expands the decoding range of the LPWAN network and
improves battery-life for nodes already in range -- allowing clients across
the network to spend less energy per transmitted bit. We implement Charm on
the LoRaWAN platform~\cite{LoRaWanAlliance2015}, a popular and widely
available LPWAN technology. Charm is implemented in a first-of-its-kind pilot
deployment for coherent diversity combining. It demonstrates increased
network coverage and improved data-rates across the clients.

\begin{figure}
    \centering
    \includegraphics[width=0.60\columnwidth]{figures/LoRaRAN.pdf}
        \vspace*{-0.1in}
    \caption{Charm: LPWAN joint decoding in the cloud}
    \vspace*{-0.1in}
    \label{fig:my_label}
    \compactimg
\end{figure}

While coherent diversity combining and PHY-layer processing in the cloud has
received much attention in the Wi-Fi~\cite{tan2009sam, xie2014scalable} and
cellular~\cite{checko2015cloud, wubben2014benefits} context, designing such a
system for low-power WANs offers radically new challenges. At the gateways, we
would have to decode very weak signals, weaker than 30 dB below the noise
floor. Simply uploading all received data to the cloud would overwhelm
the backend link, which is often  a simple home LAN. Both the LPWAN gateways
and clients are designed to be economical and deployed at scale, and without
the time synchronization required for coherent combining. At the cloud,
collating receptions from a large number of gateways at city-scale to identify
which of them contain packets from the same client is a challenge. We provide
an overview of our approach to address each of these challenges below.

\noindent \textbf{Noise-Resilience at the Gateway:} The key challenge at the
gateway is identifying packets that are significantly below the noise and
therefore virtually undetectable. A strawman approach to this problem would be
to correlate the received signal with a known preamble in any valid packet.
For instance, LoRaWAN uses a sequence of identical chirps -- signals whose
frequency increases linearly in time -- as a signature prefixing every packet.
In principle, sending an extremely long preamble could provide high resilience
to noise. In practice, doing so goes against the spirit of LPWANs where
energy for transmission is a valuable resource for the client.

Charm's approach to resolve this challenge is a hardware and software gateway
design that leverages the structure of the LoRaWAN LPWAN protocol.
Specifically, we develop a transform that converts the data symbols containing
\textit{a priori} unknown bits into a repeated and known sequence of signals
much like the preamble. Charm can therefore now use both the preamble and the
modified data sequence to detect any packet.

% adwait feb 6: seems far too complex for the intro
To understand our approach at a high-level, we present an illustrative example
that dives into the details of the LoRaWAN PHY-layer. Specifically, LoRaWAN
transmits data symbols as chirps whose initial frequency is a function of the
data. For instance over a bandwidth of 100 Hz, LoRa could represent the bit
"0" as a chirp starting at 2 Hz and bit "1" as a chirp starting at 52 Hz.
Charm's filter aliases the received LoRa signal so that frequencies modulo 50
Hz fold into each other. This means that both bit "0" and bit "1" now map to
an identical chirp starting at 2 Hz. We apply this filter through
the received packet to obtain a repeated sequence of chirps as long as the
entire packet itself. This allows us to detect the packet with a much higher
resilience to noise compared to using the preamble alone, without incurring
additional overhead.

We develop a custom gateway hardware platform integrating a Semtech LoRaWAN
radio frontend, a low-power FPGA and Raspberry PI that can filter and detect
weak signals by processing received raw I/Q samples in real-time. Our hardware
platform, a hybrid between a full SDR and a dedicated high-performance radio,
is designed to be open and highly programmable -- a novel tool to experiment
with alternative LPWAN PHY-layer designs in the 900 MHz ISM band, without
compromising on signal quality or real-time performance.

\noindent \textbf{Scalability at the Cloud:} At the cloud, Charm must deal
with a large number of receptions from various gateways in a city, pruning for
weak signals and identifying common signals between gateways. Charm proposes
multiple optimizations to run its algorithms seamlessly at city-scale. For
instance, it is often the case that gateways transmit weak signals to the
cloud for packets that have already been decoded perfectly at other gateways.
However, realizing that the weak signal has already been decoded elsewhere is
impossible without decoding it in the cloud in the first place. Charm resolves
this chicken-or-egg dilemma by exploiting both the timing and geographical
location of the received signal. Prior to sending any signal data to the
cloud, a Charm gateway sends the location, accurate timing and signal-to-noise
ratio of the received weak packet. The cloud collates such information across
multiple gateways and requests for signals only from the gateways that receive
these signals the best. In doing so Charm saves valuable uplink bandwidth at
the gateways and compute cycles at the cloud. We describe how Charm mitigates
range of other important challenges at the cloud such as imperfect timing,
frequency offsets and overlapping transmissions.

We evaluate Charm on the Carnegie Mellon University campus and around the city
of Pittsburgh, in both indoor and outdoor environments using two testbeds.
Eight user-deployed gateways built using our custom hardware platform support
a testbed covering a 0.6 sq.km. area around campus, which is used to study
Charm's performance with regard to local packet detection, range and data-rates. Four
rooftop gateways support the OpenChirp LPWAN network which services a large
10 sq.km. area that we use to acquire traces for large-scale simulations. Our
results reveal the following:

\begin{itemize}
    \item {\bf Battery-Life: }Charm improves the signal to noise ratio at a
        typical LoRaWAN client by coherently combining across 8 base stations
        by 3.16 dB, extending battery life by upto 4$\times$.
    \item {\bf Range: } We improve the maximum communication range of 8 indoor
        user-deployed gateways in urban settings from 60m in LoRaWAN to 200
        meters using Charm, an overall increase in coverage area by
        10$\times$.
    \item {\bf Coverage: } Our trace-driven simulation based on city-wide
        drive tests, estimates an overall increase in coverage by up to 2x due
        to Charm over LoRaWAN.
\end{itemize}

\noindent \textbf{Contributions:} We make the following novel contributions:
\begin{itemize}
    \item A technique that leverages the geographical diversity of unplanned
        user-deployed gateways to enable joint decoding of weak transmissions.
        This improves battery-life for users in the network and increases the
        coverage area.
    \item A hardware platform and the underlying algorithms for detecting weak
        LoRaWAN transmissions locally at the gateway.
    \item A software architecture that builds atop of LoRaWAN to enable
        joint-decoding of signals in a scalable manner.
\end{itemize}
\section{Related Work}
\label{sec:related-work}

% {\color{blue}
% [ANYTHONY, SWARUN, BOB, ARTUR, ADWAIT, DIANA, MAX]

% Particular topics to consider
% \begin{itemize}
% \item PerCom workshop paper
% \item Localization (Reverse GPS)
% \item WiFi and Cellular existing work
% \end{itemize}
% }

\noindent \textbf{Low-Power Wide-Area Networks: } Recent years have seen much
interest in LPWANs, including the development
of new hardware and standards. Private enterprises such as
Semtech~\cite{LoRaWanAlliance2015} and SigFox~\cite{sanchez2016state} have
developed LPWAN chipsets that use extremely narrow bands of unlicensed
spectrum. In contrast, cellular standardization bodies have developed two
standards for LPWAN communication for cellular base stations to communicate
with low-power IoT devices over licensed spectrum:
LTE-M~\cite{GSMAssociation2016} and NB-IOT~\cite{ratasuk2016nb}. Unlike LoRaWAN
and SigFox, these technologies require devices to periodically wake up to
synchronize with the network -- a burden on battery life.

Several recent measurement studies have been conducted to evaluate the
performance and range of LPWAN networks~\cite{petric2016measurements,
toldov2016performance} and perform theoretical capacity
analysis~\cite{mikhaylov2016analysis}. Early pilot deployment efforts are also
underway with SigFox deploying their hardware to connect security alarms to
the cloud in Spain~\cite{sanchez2016state}, smart blood refrigerators in the
Democratic Republic of the Congo~\cite{ramachandranmupnp} and smart city
applications \cite{centenaro2015long}. These efforts motivate the challenge of
limited range, performance and battery-drain of LPWAN clients. A recent
system, Choir~\cite{eletreby2017empowering},  has demonstrated improving range
and scalability of LPWANs through collaborations of weak client radios. In
contrast, this paper seeks to use collaboration between gateways without any
modifications to client behavior whatsoever to improve the battery life of
even a single client.

\noindent \textbf{Distributed MIMO and Coherent Combining: } A large body of
work has proposed the use of multiple-antennas (MIMO) to improve SNR and
reduce interference~\cite{xie2014scalable, lin2011random, kumar2013bringing}.
In the WiFi context, past systems have used multi-user MIMO to improve
performance on the uplink~\cite{shen2014rate, tan2009sam, xie2014scalable}. In
the cellular context, massive MIMO proposals have demonstrated scaling gains
of towers with a large number of antennas~\cite{shepard2012argos,
larsson2014massive}. There has been much theoretical work on distributed MIMO
overall in both the sensor networking context~\cite{del2007cooperative} and
wireless LANs~\cite{dohler2004resource} and cellular
networks~\cite{sawahashi2010coordinated}.  More recently, practical
distributed MIMO systems, primarily in the LAN-context have demonstrated both
multiplexing and diversity gains~\cite{hamed2016real, yenamandra2014vidyut,
rahul2012jmb}. Instead, our approach brings the diversity gains of distributed
MIMO on the uplink to LPWANs. In doing so we overcome multiple challenges
owing to the fact that signals at any individual tower are well below the
noise floor and are captured by low-cost hardware that lacks the precise time
synchronization required for coherent combining.

\noindent \textbf{Cloud Radio Access Networks (Cloud-RAN): } Multiple research
efforts from the industry and academia have advocated the use of PHY layer
processing at the cloud as opposed to the base stations~\cite{sabella2013ran,
hadzialic2013cloud}. In the cellular context, CloudRAN aims to perform
baseband processing at the cloud, allowing base stations to be simple and easy
to deploy~\cite{checko2015cloud, wubben2014benefits}. The key challenge
however is the need for a reliable fiber optic backhaul to the cloud to
collate data streams in a low latency manner, motivating the need for
cost-effective high-performance backhauls~\cite{liu2013case, chih2014recent}.
Our approach aims to bring PHY processing in the cloud to LPWANs that operate
at significantly lower bandwidth, with loose latency bounds and can therefore
afford Ethernet backhauls. We perform a wide variety of optimizations to
minimize the use of uplink bandwidth, including local packet detection and
data compression using an FPGA accelerator. These are helpful when the
gateways are user-deployed with residential internet backbones.
\section{Background}
\label{sec:background}

\subsection{Coherent Combining in Distributed MIMO}
\label{sec:simo}

\begin{figure}[!bht]
    \centering
    \includegraphics[height=1.25in]{figures/SIMO_cropped}
    \vspace{-10pt}
    \caption{Coherent combining helps receivers collaboratively improve
        signal-to-noise ratio}
    \label{fig:simo}
    \compactimg
\end{figure}

Wireless radios leverage multiple antennas (MIMO or multiple-input
multiple-output) to improve throughput. This paper considers coherent
combining where transmissions from a single-antenna transmitter (e.g. an
LPWAN client) are heard by multiple receiver antennas (e.g. LPWAN gateways).
These gateways can then coherently combine the received signals to improve
signal decodability.

Mathematically, let the transmitted signal be $x$ and each of the gateways
receive a signal $y_i$ through wireless channel $h_i$, introducing an
independent noise $n_i$ at the receivers. For a narrow-band system (as is
LoRaWAN and most LPWAN technologies), we can write the received signal as:
$y_i = h_i x_i + n_i $.

The receivers can now coherently combine their received signals by using the
known wireless channels $h_i$:

\compactimg

\begin{align*}
y_{\text{combined}}
	= \sum_{i=1}^N h^*_i y_i
	= \sum_{i=1}^N \left| h_i \right|^2 x + \sum_{i=1}^N h^*_i n_i
\end{align*}

The first term is the combined signal while the second term is the combined
noise. However, while the signals add up coherently, the noise, being
independent, adds up incoherently. This results in an overall increase in the
combined SNR, which allows us to jointly decode a packet, that may otherwise
not be decodable by any individual receiver.

\compactimg

\begin{align*}
SNR_{\text{combined}} %= \frac{\text{total signal power}}{\text{total noise power}} 
	= \frac{\left| \sum_{i=1}^N \left| h_i \right|^2 x \right|^2}{\sum_{i=1}^N \left| h^*_i n_i \right|^2} 
	\geq \frac{\left| \left| h_i \right|^2 x \right|^2}{\left| h^*_i n_i \right|^2} = SNR_i
\end{align*}

In practice, performing coherent combining as shown above makes two important
assumptions: (1) the packets can be detected at individual receivers above
some SNR threshold, and (2) receivers share a common clock reference for time
and frequency. This paper describes the challenges in implementing coherent
combining in the low-power wide-area context where neither assumption holds.


\subsection{Primer on LoRaWAN PHY and MAC}
\label{sec:lora}

LoRaWAN is a popular LPWAN technology that operates in the sub-GHz ISM band
(900 MHz in the U.S.) and bandwidths of 125-500kHz. LoRaWAN clients can
transmit at low-data rates (few kbps) to gateways up to 10 km away in free
space and last up to 10 years on AA batteries. Below, we detail a few key
design decisions of LoRaWAN.

\noindent \textbf{LoRa, The PHY: } LoRa's physical layer is based on
chirp-spread spectrum modulation, i.e. using a chirp signal that continuously
varies in frequency. This makes it resilient to interference, multi-path
fading and doppler effects. Every LoRaWAN packet begins with a preamble of
sixteen repeated chirps followed by data. Each data chirp encodes multiple
data bits (more precisely, \textit{chips}), with the number of  bits encoded
per chirp called the \textit{spreading factor} (SF). For instance, at
spreading factor of seven, each chirp encodes 7 bits with $2^7 = 128$ possible
uniformly separated initial frequencies. A higher spreading factor, e.g.
eight, encodes one more bit per chirp but also incurs double the transmission
time, effectively halving the data rate.\footnote{More precisely, increasing
spreading factor from $n$ to $n+1$ scales data rate by $(n+1)/2n$.} Increased
spreading factors are used to simultaneously slow down transmissions and
improve resilience to noise. LoRaWAN radios are therefore designed to transmit
at the lowest possible spreading factor that can be received at existing noise
levels for minimizing transmission time and the resulting battery drain. This
paper therefore strives to reduce spreading factor (improve data rate) for
weak transmitters.

% \vspace*{0.02in}

\noindent \textbf{The MAC: } LoRaWAN networks are designed to be simple
star-topologies that have client devices directly communicating with a gateway
that is connected to the internet over ethernet or cellular links. Gateways
are simple and relatively inexpensive forwarders that send received packets to
a cloud LoRaWAN server, and can be commanded by the server to transmit data to
clients at a specific time. Packet decoding, managing acknowledgements and MAC
parameters like data-rate are decided at a LoRaWAN server. The LoRa community
often refers to the system as having a ``MAC-in-the-Cloud'' design. LoRaWAN
allows and encourages its users to deploy their own gateways. These gateways
are completely unplanned and on low-bandwidth, unreliable internet connections
(compared to cellular base-stations that are extensively planned and have
dedicated optic fiber connections). In this paper, we refer to these as
user-deployed gateways. The penultimate goal of this paper is to make
individual unreliable user-deployed gateways more valuable by pooling together
PHY-layer processing at the cloud.

\section{Charm's Architecture}
\label{sec:arch}

The goal of Charm is to decode weak transmissions, which could not be decoded
by any individual gateway, by collating receptions from multiple gateways at
the cloud. At one level, this  enables us to expand network coverage area
reaching clients deep inside buildings, underground or in outer reaches of the
city. More fundamentally, it saves energy on the vast majority of client
devices, even if they are within range of some gateways by allowing them to
increase their data rate without experiencing any loss in performance. Our
results in Sec.~\ref{sec:energy-savings} demonstrate that lowering transmit
time results in a direct and significant impact on battery life.

Fig.~\ref{fig:architecture} depicts Charm's architecture where we assume the
gateways can be user-deployed  both indoors and outdoors, at a cost of a few
hundred dollars. These base stations have an Ethernet backhaul to the cloud
that accommodate a maximum uplink bandwidth of a few megabits per seconds.
Much like the standard LoRaWAN architecture, MAC-layer scheduling is performed
at the cloud with gateways relaying their received data to the cloud. However,
to accommodate decoding weak received signals, we also allow gateways to ship
raw received I/Q signals from feeble low-power clients to the cloud. The cloud
aggregates such weak signals and coherently combines them to decode the
underlying data bits from feeble receptions across multiple gateways. In other
words, Charm performs a joint optimization of the  PHY-layer at the cloud,
simultaneously improving battery life and range of low-power clients at the
expense of increased computation at the cloud.

Realizing a scalable and real-time system based on the above architecture
is challenging both at the gateways and the cloud:
\begin{itemize}
\item {\bf At the Gateway: } Given that signals from weak LPWAN
clients are often well below the noise floor, gateways are unaware of these
packets in the received signal. This means that base stations must effectively
send all their received raw signal data to the cloud to detect and decode weak
signals, stressing their limited uplink bandwidth.
\item {\bf At the Cloud: } The cloud must identify signals from which
gateways need to be combined to recover transmitted data from multiple
clients. At city-scale, it is conceivable that overlapping weak transmissions
from different clients are received at the same time by gateways, making data
recovery challenging at the cloud. Additionally due to the use of low-cost
hardware that lacks precise time synchronization, each of the gateways adds
clock and frequency errors to the captured signals. These must be resolved
before the signals can be combined.
\end{itemize}

\begin{figure}[!htb]
    \centering
    \includegraphics[width=0.45\textwidth]{figures/charm-architecture_cropped.pdf}
    \caption{Architecture of Charm}
    \label{fig:architecture}
    \compactimg
\end{figure}

The rest of this paper describes Charm's solutions to each of these
challenges. Specifically, Charm makes two key contributions: (1) A software
interface at the gateway to identify weak transmissions to ship to the cloud,
and a hardware design that facilitates these decisions in real-time; (2) A
scalable cloud based PHY-layer processing system at the cloud that can operate
at city-scale. Next we elaborate on each of these components.
\section{The \name\ Gateway}
\label{sec:gateway}

% {\color{blue} Points to cover
% \begin{itemize}
%     \item Hardware + block diagram + capability + how to decode IQ streams
%     \item Local detection algorithm
%     \item Optional: (LoRa-aware) compression
% \end{itemize}

% }

We first describe \name's design at the gateway to enable accurate decoding of weak clients, by relaying suspected weak signals to the cloud. \name\ achieves this first through a software algorithm at the gateway that identifies weak transmissions that may be significantly below the noise floor. We further implement this approach in hardware by building a custom programmable radio platform for the gateway, that streams and processes raw I/Q samples using an FPGA. We show how a \name-gateway can detect weak signals in real-time through this design, while simultaneously being programmable and responding to policy changes from the cloud. 

\subsection{Locally Detecting Weak Signals}
\label{sec:local-detection}
To reap the benefits of coherent diversity combining across multiple gateways, \name\ must relay weak signals to the cloud. Unfortunately, many -- perhaps all of these signals may be well below the noise floor at the gateway by as much as 30~dB, in which case we lose the benefits of coherent combining. Yet, transmitting all received signals to the cloud to overcome this problem is unfeasible given that gateways have limited uplink bandwidth to the cloud. To put this in perspective, streaming all received I/Q samples to the cloud requires an uplink bandwidth of 72 Mbps to the cloud. However, the vast majority of LP-WAN gateways are likely to be user-deployed hardware such as set-top boxes that cannot afford this bandwidth. Indeed, this creates  trade-off between detecting weak transmitters and conserving up link bandwidth.

\name\ breaks this trade-off by detecting weak signals well below the noise floor at a single LP-WAN gateway. At a high level, our solution relies on the structure of the LoRaWAN protocol. Specifically, LoRaWAN transmits signals in the form of chirps, i.e. signals whose frequencies increase linearly in time. In addition, several of these chirps are identical. For instance, consider the initial preamble  in LoRaWAN with as many as 16 identical and consecutive chirps. This means one can design a receiver that coherently sums up adjacent symbols of any received signal over a sliding window. If the summing-up operation is truly coherent, the underlying signal (i.e. the chirp) will add up constructively, while noise will add up incoherently. In effect, this boosts the signal-to-noise ratio of the received signal significantly, allowing us to detect at least the preamble of a LoRaWAN packet. One can then deliver a long chunk of packets surrounding this preamble to the cloud. 

However the resolution of the above approach is a function of preamble length -- the longer the preamble sequence is, the greater will be the extent of noise that \name\ can tolerate. Transmitting extremely long preambles increases the overhead of the communication system, and in the long term, impacts battery life. \name\ therefore develops an approach that can detect weak signals by leveraging data symbols in addition to the preamble -- even though the transmitted data sequence is unknown a priori at the gateway. We detail our approach below. \vspace*{0.1in}

\noindent \textbf{Leveraging the structure of LoRaWAN data: } \name\ seeks to use the structure of the data symbols in LoRaWAN to improve detection of the packet in the presence of noise. Indeed, much akin to the preamble, the data symbols of a LoRaWAN packet are also composed of a sequence of chirps. Unlike the preamble though, LoRaWAN data is composed of a sequence of chirps with different frequency-shifts based on the bits they represent. Assuming that the underlying data in a message is completely unknown and arbitrary, this makes looking for structure within the data challenging. 


\begin{figure}
    \centering
    \includegraphics[width=0.45\textwidth]{figures/CharmEnhancedDetection.pdf}
        \vspace*{-0.1in}
    \caption{Enhanced Charm Packet Detection}
    \label{fig:enhanced_charm}
    \compactimg
\end{figure}

\name\ relies on the fact that while the data does cause shifts in frequencies of chirps within the packet -- these shifts are not completely random. In particular, chirps can undergo a discrete number of possible shifts based on the number of bits per chirp. For a spreading factor of $SF$ (i.e. a transmission data rate of $SF$ bits per chirp), the frequency shift is one of $2^{SF}$ values. \name\ therefore implements a solution that coherently reinforces adjacent chirps, modulo the minimum possible frequency shift between them. This ensures that regardless of their underlying data, adjacent chirps always add up to reinforce each other while noise adds up destructively as before. Given that there are a significantly larger number of data symbols when compared to preamble symbols in any transmission, this provides an additional mechanism to detect packets below the noise. 

Mathematically, let $y_1, y_2, \dots, y_m$ denote the $m$ received data symbols and $x_1, x_2, \dots, x_m$ denote the transmitted data bits encoded as frequency shifts, each a number between $0$ and $2^{SF-1}$ where $SF$ is the spreading factor. Let $\delta f = \text{Bandwidth} / 2^{SF}$ denote the minimum possible frequency separation between two encoded data chirps.  Then we can write the received signal at any time $t$ of the $i^{\text{th}}$ symbol as:
\begin{align}
    y_i(t) = h e^{j 2 \pi (f(t) - x_i \delta f) t} + n_1 \label{eqn:yi}
\end{align}
Where $f(t)$ denotes the time varying frequency of the chirp, $j$ is the square root of $-1$, $h$ represents the wireless channel and $n_i$ represents noise. 

When multiplied by $e^{-j 2 \pi f(t) t}$ and viewed in the Fourier domain, this results in a single tone at frequency $x_i \delta f$ subject to noise. Clearly the location of the tone is a function of the underlying data -- a different quantity for different data symbols. 

In contrast, let us sub-sample the above equation at times $t$ that are multiples of $1/\delta f$ (let's say $t = \frac{k}{\delta f}$ for integer values of $k$). 

\begin{align}
    y_i\left(t\right) = h e^{j 2 \pi (f(t) - x_i \delta f) \frac{k}{\delta f}} + n_1 = h e^{j 2 \pi f(t) t} + n_1\label{eqn:yi2}
\end{align}
This time, when multiplied by $e^{-j 2 \pi f(t) t}$ and viewed in the Fourier domain, this results in a single tone at frequency $0$ (subject to noise) regardless of the underlying data in each symbol. In other words, sub-sampling in the time domain led to aliasing of all the data peaks in the frequency domain into one frequency bin (in this case, the DC bin), while noise is smeared uniformly across all bins. Indeed, \name\ repeats the sub-sampling across multiple time steps separated  by $\frac{1}{\delta f}$ and averages the results. The resulting average reinforces peaks corresponding to all the data symbols coherently in one Fourier frequency bin, while noise adds up incoherently among all remaining bins. This leads us to a very natural LoRaWAN packet-detection mechanism that applies this operation across different sliding windows of the received signals. We signal the presence of a packet once our algorithm detects a significant peak in the Fourier domain that dominates other peaks (subject to a threshold). Given that our approach averages results over a large number of data symbols, it remains resilient to noise without making assumptions about the contents of the packet itself.  

\RestyleAlgo{boxruled}
\LinesNumbered
\begin{algorithm}[ht]
\caption{Charm's enhanced detection algorithm}
\label{alg:algorithm-label2}
 \For{bits in instream}{
 [C=I+jQ]=downsample(bits)\;
 \For{chirp\_length in C}{
    F=chirp\_length$*$down\_chirp\;
    FCollect.collect(F)\tcp*{Data Collection} 
 }
 C=FCollect.modulo($\delta f$)\tcp*{Modulus Bucketing}
 \If{$\frac{max(abs(fft(C)))}{mean(abs(fft(C)))}>\tau$}{
    SEND C to CLOUD \tcp*{Packet Forwarding}
 }
 }
\end{algorithm}

\noindent \textbf{Mitigating Frequency Offsets: } To add up signals from adjacent symbols coherently, \name\ must assume that the received signals in these signals are identical -- subject to noise and discrete shifts in frequency due to the data (as described above). In practice however, wireless signals from the LP-WAN client to the gateway experiences an additional arbitrary shift in frequency due to Carrier Frequency Offset (CFO). CFO stems from the subtle variation in frequency between the clocks on the transmitter and receiver. Given that the client is inexpensive, its clock often exhibits large and arbitrary frequency differences relative to the gateway. 

Two properties of CFO make its impact on \name's algorithm above particularly damaging: (1) First, CFO unlike data introduces a frequency shift that is not discrete but continuous. As a result it is not simply eliminated by looking at the chirp in the Fourier domain ``modulo $\delta f$''  akin to the data as described above. (2) Second, CFO introduces a continuous phase shift $2 \pi \Delta f_{CFO} t$ onto the received signal that accumulates over time. This means that even otherwise identical received symbols may add up incoherently owing to a time-varying phase shift. 

The straw man approach to eliminate CFO would be an attempt to directly estimate it. For instance, one could rely on the repeated symbols of the preamble where any phase variation is purely a function of CFO. For instance, the phase shift between two otherwise identical preamble symbols separated by $t$ is simply $2 \pi \Delta f_{CFO} t$, which can be used to solve for $\Delta f_{CFO}$ and eliminate its effect. This solution works well if the transmission truly contains a sufficient number of preamble symbols to overcome the impact of noise. However, it precludes us from using the large number data symbols akin to our approach above to greatly enhance our resilience to noise. 

\name\ overcomes this problem by realizing that while estimating $\Delta f_{CFO}$ from the data symbols alone is challenging, it is sufficient to estimate $\Delta f_{CFO}$ modulo $\delta f$ to detect the LoRaWAN packet. To see why, recall that the frequency offset over a packet $\Delta f_{CFO}$ can be decoupled into two components: $[\Delta f_{CFO}]$ an integer multiple of  $\delta f$ and $\{\Delta f_{CFO}\}$, the  fractional component modulo $\delta f$. When looking at the data chirps in the frequency domain modulo $\delta f$, all the data symbols appear identical given that all frequency shifts of the data are a multiple of  $\delta f$. Indeed, the integer part of the CFO: $[\Delta f_{CFO}]$, suffers the same fate as the data and does not introduce any perceivable changes between the symbols whatsoever. Only the fractional part of the CFO: $\{\Delta f_{CFO}\}$ persists and introduces a time varying phase shift $2 \pi \{\Delta f_{CFO}\} t$ across symbols. This means that we can simply solve for $\{\Delta f_{CFO}\}$ and eliminate its effect akin to the straw man approach, this time on the data symbols in the frequency domain modulo $\delta f$. In other words, \name's solution remains resilient to frequency offset, both in detecting the preamble as well as data symbols of a LoRaWAN packet. 

% WHy is it necessary?

% Why is it hard?

% How do we implement it?

% How well does it perform?

\subsection{Programmable Hardware Design}
\label{sec:hardware}

\begin{figure}[!htb]
\centering
\begin{tabular}{@{}c@{}}
\subfloat[Block diagram of the \name\ programmable radio hardware platform]{\includegraphics[width=0.40\textwidth]{figures/lpran-block_cropped}
\label{fig:lpran-block}} \\
\subfloat[\name\ programmable GW]{\includegraphics[width=0.23\textwidth, height=1.2in]{figures/gw-anon-sm}
\label{fig:gw-pcb}} 
\subfloat[Gateway with \name\ board and LoRaWAN~concentrator]{\includegraphics[width=0.23\textwidth, height=1.2in]{figures/annotated_gateway}
\label{fig:gw-annotated}}
\end{tabular}
\vspace*{-0.1in}
\caption{\name\ Hardware Platform}
\label{fig:lpran-hardware-images}
\compactimg
\end{figure}

\name\ must process raw I/Q samples from the gateway and selectively relay this information to the cloud in real-time. However, existing LoRaWAN gateway hardware cannot provide the raw I/Q streams necessary for joint decoding. In contrast, deploying a full software-defined radio (SDR) at the gateway allows packet decoding, it comes with high cost in term of power, sensitivity and unit price. We therefore develop a custom \name\ hardware platform shown in \figref{lpran-hardware-images} as an auxiliary peripheral to a gateway and can provide the necessary quadrature streams. Key to our performance is a light-weight, low-cost and easy-to-reprogram hardware accelerator for data reduction enabling further local processing (e.g. on the accelerator or by a Raspberry Pi). In effect, we allow for a system that simultaneously allows some SDR-like programmability of the PHY while maintaining high performance and low cost. \vspace*{0.1in}

%We includes a LNA in our front-end for enhanced sensitivity and match the performance of the current gateway hardware. Noted, the gain is independent of LNA; thus, better LNA would results in better gain. 

\noindent  {\bf Compressing the Data Stream: } The raw IQ stream would be too much for a low-power microprocessor, and  also contain too much redundant information for our purpose. In particular, we use the SX1257 RF front-end that provides 1-bit delta-sigma modulated signals at a whopping 36 MSps each for the I and Q streams.  In order to keep the data stream to a more microprocessor-friendly load, the design would require some lossless compression. Through careful choice of parameters, we chose to compress the IQ stream by summing consecutive samples in windows of size 64 and convert it into a single 7-bit sample: 
\begin{equation}
x_i = \sum_{j=64*i}^{64*i + 63} s_j
\end{equation}
, where ($x_i$) is the analyzable samples, and ($s_j$) is the I/Q sample rates. A window size of 64 is selected since we are only interested in a final
bandwidth of approximately 500 kHz that the RF front-end is capable of
capturing. Upon applying the above technique, the compressed I/Q streams generate data at a rate of 1.125 MBps, down from the original 72 Mbps. \vspace*{0.1in}


\noindent  {\bf Programmability: } The compressed I/Q samples are stored locally on a Microsemi IGLOO AGL250 FPGA. The low-cost low-power FPGA is a hardware accelerator, which allows local processing (e.g. down-sampling, preamble detection, synchronization, etc.) and necessary data reduction.  The block of samples are then double buffered to ensure the validity of the data during transfers. The stream of data is transferred using a high-speed serial interface to the microprocessor to relay back when requested by the joint-decoder or additional processing. In particular, we transfer data to a Raspberry Pi over an SPI bus, which can then perform additional local processing, time-stamping and temporary local storage until a stream is requested by the joint-decoder. While we are not intending our hardware platform to be a full-scale SDR (given that it only supports LoRaWAN frequency), the FPGA allows programmers to implement advanced real-time algorithms for packet decoding and/or full duplex transmission across multiple channels. In addition, the Raspberry Pi allows for ease of programmability when gathering low-rate statistics about the received signals at the gateway. Overall, we believe the \name\ hardware platform will reduce the barrier for LP-WAN PHY-layer innovation for programmers and researchers across the board. 

% (Swarun: Don't understand this sentence) Noted, the number of samples are buffered in each block of samples depends on the performance of the microprocessor and the serial transfer rate.

% {\color{blue} Describe capabilities of the platform. How much bandwidth supported, size of streams, how is large is the output and how do we handle it. Also describe what else it can be used for.}

% Implementation details

% {\bf Compressing the Data Stream: } With this in mind, we pairs a Semtech SX1257 868/900 MHz RF front-end with LNA \cite{sx1257} and a Microsemi IGLOO
%AGL250 FPGA. The low-cost low-power FPGA is just the hardware accelator, which allows local processing and necessary data reduction. The processed data streams are then transferred to a Raspberry Pi over an SPI bus, which can then perform additional local processing, time-stamping and temporary local storage until a stream is requested by the joint-decoder. The SX1257 RF front-end provides 1-bit delta-sigma modulated signals at 36 MSps for the I and Q streams. 
%A window size of 64 is selected since we are only interested in a final
%bandwidth of approximately 500 kHz that the RF front-end is capable of
% capturing. Applying this technique, the uncompressed I/Q streams generate data
% at a rate of 1.125 MBps. While we are not intenting the hardware to be a full SDR, the FPGA also allows user to implementing advance algorithm for packet decoding and/or full duplex transmission with multiple channels.



% \name\ must process raw I/Q samples from the gateway and relay this information to the cloud in real-time. However, existing LoRaWAN gateway hardware cannot provide the raw I/Q streams necessary for joint decoding. In contrast, deploying a full software-defined radio at the gateway allows packet decoding, it comes with high cost in term of power and unit price. We therefore develop a custom platform, named \textit{LPRAN} and shown in \figref{lpran-hardware-images} as an auxiliary peripheral to a gateway and can provide the necessary quadrature streams. Our architecEven though the full SDR allows packet decoding, it comes with high cost in term of power and unit price. hardware has to scale according in term of performance to keep up with the data stream. Thus, we focused on two key design factors: 1. high-performance RX, 2. hardware accelerator for data reduction. 

% We includes a LNA in our front-end for enhanced sensitivity and match the performance of the current gateway hardware. Noted, the gain is independent of LNA; thus, better LNA would results in better gain. 

% The raw IQ stream would be too much for a low-power microprocessor, also unnecessary for our purpose. In order to keep the data stream to a more microprocessor-friendly load, the design would require some lossless compression. Through careful thoughts and designs, we chose to compress the IQ stream by sum consecutive samples in windows of size 64 and convert it into a single 8-bit sample
% \begin{align*}
% x_i &= \sum_{j=64*i}^{64*i + 63} s_j
% \end{align*}
% , where ($x_i$) is the analyzable samples, and ($s_j$) is the IQ sample rates. 
% These samples is stored locally on FPGA for further processing (e.g. downsample, preample detection, sync, etc...). The block of samples are then double buffered to ensure the validity of the data during transfers. The stream of data is transferred using high-speed serial interface to the microprocessor to relay back when requested by the joint-decoder or additional processing. Noted, the number of samples are buffered in each block of samples depends on the performance of the microprocessor and the serial transfer rate.

% {\color{blue} Describe capabilities of the platform. How much bandwidth supported, size of streams, how is large is the output and how do we handle it. Also describe what else it can be used for.}

% % Implementation details
% With this in mind, we pairs a Semtech SX1257 868/900 MHz RF
% front-end with LNA \cite{sx1257} and a Microsemi IGLOO
% AGL250 FPGA. The low-cost low-power FPGA is just the hardware accelator, which allows local processing and necessary data reduction. The processed data streams are then transferred to a Raspberry Pi over an SPI bus, which can then perform additional local processing, time-stamping and temporary local storage until a stream is requested by the joint-decoder. The SX1257 RF front-end provides 1-bit delta-sigma modulated signals at 36 MSps for the I and Q streams. 
% A window size of 64 is selected since we are only interested in a final
% bandwidth of approximately 500 kHz that the RF front-end is capable of
% capturing. Applying this technique, the uncompressed I/Q streams generate data
% at a rate of 1.125 MBps. While we are not intenting the hardware to be a full SDR, the FPGA also allows user to implementing advance algorithm for packet decoding and/or full duplex transmission with multiple channels.

\section{Charm in the Cloud}
\label{sec:cloud}

At the cloud, Charm seeks to coherently combine received signals from multiple
gateways to recover weak received signals. At a high level, Charm collates I/Q
samples from multiple gateways and estimates their packet start time and
wireless channel. It then uses standard coherent SIMO combining (see
Sec.~\ref{sec:simo}) of the same weak transmission across multiple gateways to
ensure that the data can be accurately recovered. Charm repeats this
cloud-based PHY-layer processing at city scale across clients and gateways.

The rest of this section describes the key challenges and opportunities in
making the above design scalable and practical. First, we describe Charm's
approach to ensure accurate time-synchronization between gateways -- showing
how even an offset of one or two samples can be severely detrimental to
coherent combining. Second, we present our solution to dynamically infer
signals from which gateways need to be combined over time to best recover a
weak signal. Finally, we present opportunities to improve bandwidth and system
performance at the cloud by avoiding wasted transmissions of I/Q data to the
cloud as well as wasted computation.

\subsection{Time Synchronization at the Cloud}
Charm relies on the accurate timing of received weak signals at the gateways
for two important reasons: First, any offset in timing between signals
corresponding to the same packet across  gateways will prevent the signals
from coherently combining. Second, the precise start time of packets across
gateways is valuable information to identify the packet, allowing Charm to
infer which received signals across gateways correspond to the same packet.

A naive approach to synchronize base stations would be to synchronize them
through highly accurate clocks (GPS-synced) or through time-synchronization
protocols in software over the backbone network (e.g. NTP). In practice, for
indoor gateways (e.g. set top boxes) connected to an Ethernet backhaul, these
can provide time synchronization of up to a few milliseconds. In practical
terms, this means that the received signals at the gateways can be time
synchronized to within a small number of time samples.

\begin{figure}
    \centering
    \includegraphics[width=0.45\textwidth, height=1.6in]{figures/TimeOffset.pdf}
    \vspace*{-0.1in}
    \caption{Effect of timing offset on detection}
        \vspace*{-0.0in}

    \label{fig:toffset}
    \compactimg
\end{figure}

Unfortunately, even a small offset in the timing between two gateways can
severely deteriorate coherent combining. Fig.~\ref{fig:toffset} depicts a
simple example of the phase difference between two gateways whose signals are
offset by zero and one sample respectively. We note that even an offset of one
frequency bin causes a significant time-varying error in phase between the
gateways. As a result, summing up these signals would cause some symbols
in-phase to reinforce, while others that are out-of-phase cancel each other.
\vspace*{0.1in}

\noindent {\bf Phase-Based Time-Sync Below the Noise: } Charm overcomes this
challenge by recognizing that small time-errors between two gateways results
in a phase difference over time that is predictable. As shown in
Fig.~\ref{fig:toffset}, this phase difference is a linear function of time,
given by $2\pi f(t) \delta t$, where $f(t)$ is the instantaneous frequency of
the chirp (linear in time) and $\delta t$ is the required timing offset. In
principle, one can therefore estimate the slope in phase over time to recover
the timing offset. In practice however, doing so is challenging, particularly
when each received signal at each gateway is completely buried below the
noise. The phase of such signals at any such gateway simply appears to be
random -- making any form of linear regression of the slope highly
error-prone.

Charm overcomes this challenge using two key properties: First, owing to
coarse time synchronization of the gateways (via NTP), any residual timing
error between them is limited to a few samples. This allows Charm to
iteratively optimize over a small number of time-shifts to infer the offset
that leads to the best fit. Second, Charm's can extract timing offsets both
from the preamble and the data symbols. To see how, notice that our approach
only considers the {\it difference} in phase between the same packet heard at
two different gateways. Given that, in the absence of timing offsets, both
gateways perceive the same underlying message bits over time, the resulting
phase difference would be independent of the transmitted data bits -- whether
they belong to the preamble or data.

Charm's approach therefore considers a the range of possible small offsets
between any two received signal sequences. For each candidate offset, it
computes the phase difference between the signals as a function of time. It
then identifies the true offset between the gateways as the one whose phase
difference varies minimally across the entire packet. Given that our approach
averages measurements through the entire packet (both preamble and data), it
remains highly resilient to noise.\vspace*{0.1in}


\noindent {\bf Maintaining Synchronization across Packets: } Finally, Charm
can learn the time-offsets between gateways, particularly in busy urban
deployments, by using information from past packets. Recall that Charm's
coherent is only affected by timing errors between pairs of gateways -- not
the gateway and any particular client. While these errors may change over
time, over small intervals (e.g. hundreds of milliseconds), they are unlikely
to change. As a result, one can use the measured time offset from a previous
packet to infer the offset at the next packet that follows soon after. This
allows us to maintain a history of the time-offsets, smoothed by algorithms
such as Kalman filtering with outlier rejection, that helps us better predict
time offsets between gateways even when signals from certain clients are too
weak to measure these reliably.

\subsection{Joint Decoding at the Cloud}
\label{sec:joint-decoding-cloud}
This section answers an important question: How does Charm decide which weak
signals received from a set of gateways need to be combined coherently? In
other words, Charm must identify which signals at the gateway correspond to
the same packet from the same transmitter. It must do so even in the presence
of overlapping transmissions from multiple clients at geographically different
locations.

\noindent {\bf Who should we combine?: } Charm addresses this challenge by
using the timing information of packets to infer transmissions that correspond
to the same user. It further uses the perceived signal-to-noise ratios and
geographic location of the gateways and measures the likelihood that far-away
gateways can listen to transmissions from the users at the observed
signal-to-noise ratios. It then calculates a feature vector for each received
signal that contains two tuple: (1) The time instance at which the packet was
received; and (2) The geographic location of the gateway. We apply the OPTICS
clustering algorithm~\cite{ankerst1999optics} to then cluster received signals
from multiple clients at any time instance.

Past-clustering, we combine received signals from a subset of clients in each
cluster. Specifically, we only choose to combine signals with a sufficiently
high signal-to-noise ratio. This is because transmissions that are highly
noisy tend to add little additional value yet cost uplink bandwidth.

An important consequence of our clustering approach based on geographic
location of the gateway is that it facilitates spatial re-use. Specifically,
it is quite possible that weak transmissions from two different neighborhoods
occur at the same time but are heard at distinct subsets of gateways. Charm
allows us to decode these transmissions simultaneously without mixing up their
signals. Indeed, gateways that are geographically in-between and hear
interfering signals from both clients can be simply weeded out from clustering
due to their poor signal-to-noise ratio.

\noindent {\bf Joint Decoding Algorithm: } Algorithm~\ref{alg:algorithm-label}
below describes Charm's joint-decoding algorithm end-to-end. At a high level,
our approach retrieves the wireless channels of the signals to be combined at
any instance, their timing offsets and frequency offsets computed as described
in the above sections. We, then eliminate any phase errors owing to time and
frequency offsets in the received signals. We then coherently sum up the
resulting signals multiplied by the conjugate channels as described in
Sec.~\ref{sec:background}.

\RestyleAlgo{boxruled}
\LinesNumbered
\begin{algorithm}[ht]
\caption{Joint decoding algorithm}
\label{alg:algorithm-label}
 packets = receive\_data(candidates)\;
 \For{p in packets}{
 p = $e^{j2\pi (\Delta_f)t}$ p \tcp*{Freq Offset Correction}
 p = $e^{j2\pi f(\Delta_t)}$ p \tcp*{Timing Offset Correction}
 h(p)$=\frac{p}{reference}$ \tcp*{Channel Estimation}
 }
 combined\_packet$=$zeros(p)\;
 \For{p in packets}{
 combined\_packet = combined\_packet + h$^*$p \;
 }
 decode(combined\_packet)\;
 SEND ACK\;
 
 \end{algorithm}

\subsection{Opportunistic Fetching of Information}
Our design thus far assumes Charm gateways relay raw I/Q received signals to
the cloud, only if their signals are too weak to be decoded, yet can be
detected. However, this approach can be ineffective for two reasons: (1) On
the one hand, the cloud may have already received the decoded data bits from
another gateway, meaning that Charm simply wasted uplink bandwidth
unnecessarily; (2) On the other hand, some received signals may be
significantly below the noise floor even to be detected, yet be valuable
enough to be relayed to the cloud to be jointly decoded with other such weak
receptions.

\noindent {\bf Two-Phase Data Fetch: } Charm overcomes these challenge through
a pull based approach where gateways relay raw I/Q samples to the cloud, only
when explicitly asked for by the cloud. Each gateway keeps a circular buffer
of I/Q streams as well as any recent snapshots containing a potential packet.
For each potential reception, a gateway first reports its signature (center
frequency and spreading factor), the time of the reception packet, the
perceived wireless channel and signal-to-noise ratio. Charm then performs
clustering as described above and requests the raw I/Q samples {\it only }
from clients whose signals were chosen to be combined. Given that latency to
the cloud are of the order of few milliseconds, smaller than a typical LoRaWAN
packet size (tens, often hundreds of milliseconds), our system can perform
decoding virtually in real-time at LPWAN timescales, despite incurring
multiple round-trip times in fetching information.

\noindent {\bf Opportunistic Data Buffering: } In some instances, Charm's
clustering algorithm may fail to have enough signals to successfully combine
and decode a packet using the gateways that detected the packet alone.
However, Charm may be able to opportunistically fetch information from other
gateways in the same geographical region of the cluster and tuned to the same
frequency who may have received the same signal, yet at a signal-to-noise
ratio too weak to detect locally. Charm therefore requires all gateways to
store past signals for up to 1.6 seconds (maximum LoRaWAN packet length) in
the past in a 5 MB circular buffer. This allows Charm to query and fetch
signals from gateways, even in scenarios where only one gateway in the entire
network was able to locally detect a signal from a given transmitting client.
\section{Integration with LoRaWAN}
\label{sec:implementation}

\begin{figure}
    \centering
    \includegraphics[width=0.45\textwidth]{figures/deployment.pdf}
    \caption{Deployment photos and coverage heat maps (\textit{Parts of images have been occluded to maintain anonymity})\cmt{}{Change the image to complete one}}
    \label{fig:deployment}
\end{figure}

Charm is implemented as a service running on a campus-wide LoRaWAN network installed at anonymous University.  We currently have four gateways mounted on rooftops providing wide area coverage and eight auxiliary indoor gateways extending coverage into remote parts of campus. The LoRaWAN network is powered by the open-source Anonymous framework that allows students and faculty to login with their campus accounts and create device endpoints for capturing and sharing data.  Anonymous provides services that can be attached to data streams that can perform operations ranging from basic data storage to binary-to-JSON packing and unpack. A RESTful interface is used to configure meta-information about devices and set access control privileges that define how other users can interact with data streams.  The only modifications required to make a gateway Charm enabled is the additional hardware platform for receiving raw I/Q streams and a modified LoRaWAN packet forwarder that runs the packet reception event detector, maintains a circular buffer of I/Q streams and brokers interactions with the Charm cloud.  Communication between gateways and the cloud is managed using the Anonymous's MQTT publish subscriber messaging layer where compressed Charm packets can be easily grouped and organized based on location.  The Charm service can instruct clients to switch to faster data rates (as compared to the normal data rate negotiation process) by spoofing improved SNR values during the join process. In this way, Charm can seamlessly operate with existing LoRaWAN devices with no modification.

\figref{deployment} shows examples of our gateway hardware deployed in the field along with the coverage in and around campus.  The network is currently supporting a wide-range of applications from student projects, study-space monitoring, and building occupancy sensing all the way to mechanical room environmental sensing and utility sub-metering for the campus facilities maintenance team.  The client transmitters in our experiments use the Semtech SX1276 LoRaWAN chipset. The figure also shows an example coverage heat map generated by nodes deployed throughout campus and the neighboring area.  We see that the network with just four outdoor gateways is able to cover almost 10$km^2$ of urban space.





\section{Results}
\label{sec:eval}

%{\color{blue} }
%We implemented our system using 8 LPWAN boards as base stations distributed across a university campus.  We used a Semtech SX1276 LoRaWAN transmitter as the client device. We use Charm as the software platform at the backend to support backhaul. 

We evaluate Charm both through proof-of-concept experiments and large-scale
trace-driven simulations. We perform our experiments in various indoors and
outdoors environments across the campus.

\subsection{Role of Transmission Rates on Battery Life}
\label{sec:energy-savings}

We study the energy profile of a typical battery-operated LoRaWAN client, as
in \figref{power-trace}. The device performs some local computation, sends a
LoRa message, waits for an acknowledgement and then goes to low-power sleep
mode. The radio transmission consumes the highest amount of energy ($=
\text{area under the curve} \times \text{voltage}$) by a large margin. Thus,
any optimization to battery life must focus on reducing the energy of
transmissions.

Two parameters affect the energy consumed by transmissions: (1) transmit power
and (2) transmit time. Using the currently available LoRa radio chipsets
(Semtech SX1272 and SX1276), we've observed that the transmit power does not
significantly change the power drawn from the battery during transmission. Any
optimization will thus have to focus on reducing the transmit time. The
transmit time is determined by the data rate and the amount of data to send.
We do not control the amount of data generated by client devices and thus,
improving the data rate would provide the largest improvements.

\figref{lifetime-estimates} shows the estimated battery life of a client
device if it were to communicate with different data rates. Wireless systems
try to communicate at the highest data rate that does not cause too many
errors. In the case of LoRa devices, switching to a slower data rate increases
the spreading factor,  which have better sensitivity on the receiver. Thus,
LoRa devices communicating at the highest spreading factors (and
correspondingly using the lowest data rates) can communicate at much longer
range and with higher reliability. The downside is a significant increase in
their transmission time which severely affects battery life. This demonstrates
that Charm can significantly improve battery life should it allow clients to
transmit at higher data rates.

Charm can not only increase range but also increase coverage in urban
scenarios with lots of obstructions and devices deep inside buildings.
\figref{penetration-test} shows a penetration test experiment inside an
on-campus poured-concrete building. Despite a gateway being placed on the roof
of the building, we observe the received signal strength to vary as much as 46
dBm at various locations inside the building. A number of client devices, deep
inside structure, would have been forced to use the the slowest data rate but
can now benefit from Charm.

\begin{figure}[!t]
\centering
\compactimg
\begin{tabular}{@{}c@{}}	
\subfloat[Typical client device current consumption for a complete LoRaWAN transmission. The device is powered at $3V$.]{\includegraphics[width=0.36\textwidth]{figures/bug_power_trace_annotated}
\label{fig:power-trace}}\\
\hspace*{0.1in}
\subfloat[Estimated lifetime of a client device powered by two AA batteries sending 36 byte packets at various data rates based on the energy profile.]{\includegraphics[width=0.36\textwidth]{figures/LoRaBug_AA_lifetime_semilog}
\label{fig:lifetime-estimates}}
\end{tabular}
\vspace{-10pt}
\caption{Power statistics}
\compactimg
\end{figure}

\subsection{Local Detection Algorithm}
\label{sec:local-detection-eval}

We performed trace-driven simulation to demonstrate the increase in detection
capability of LoRa packets under noise. To perform this experiment, we collect
data at different spreading factors at high SNRs. We then measure the signal
power and progressively increase additive white Gaussian noise in the signal.
At every dB of decrease in SNR, we test the state-of-the-art LoRaWAN decoding
algorithm against Charm's local and enhanced detection algorithm, where the
former uses the preamble alone and the latter uses both preamble and data in
its optimization.

\begin{figure}[!t]
\centering
\vspace{-20pt}
\subfloat[Success rate for bi-directional packet exchange between client device and gateway]{
\includegraphics[width=0.45\columnwidth]{figures/penetration_wean_success_cropped}
\label{fig:pen-wean-success}}
\hfill
\subfloat[RSSI at the gateway for successful transfers]{
\includegraphics[width=0.45\columnwidth]{figures/penetration_wean_rssi_cropped}
\label{fig:pen-wean-rssi}}
\vspace{-10pt}
\caption{RF signal penetration experiments in a large poured-concrete building}
\label{fig:penetration-test}
\compactimg
\end{figure}

The results in \figref{local-detection} show that Charm's local detection
algorithm far outperforms the LoRaWAN detection algorithm for detection of the
packet. Further, Charm's enhanced algorithm outperforms Charm's local
detection algorithm by up to 10 dB, as it uses data symbols in addition to the
preamble, improving performance. This demonstrates the value of using data
symbols to better optimize packet detection, particularly in noisy settings.
Our results also reveal a 33\% increase in the negative SNR under which a
packet can be detected, when compared to LoRaWAN -- a gain of between 16-30
dB. To put this in perspective, this is equivalent boost in SNR by coherent
combining of between 40-1000 gateways. Said differently, Charm can even detect
packets that can only be decoded by coherently combining signals from at least
40 gateways that receive the same signal at a similar level of noise.

\begin{figure*}[!t]
\centering
\subfloat[Local packet detection capability for low SNR receptions. Methods using data and preamble symbols outperform preamble-only methods.]{
\includegraphics[height=3.5cm]{figures/local_detection_limits}
\label{fig:local-detection}}
\hfill
\subfloat[Diversity gain: Combined SNR increases logarithmically as more gateways recieve a transmission]{
\includegraphics[height=3.5cm]{figures/diversity_gain}
\label{fig:diversity-gain}}
\hfill
\subfloat[Client battery life improves as a larger number of receiving gateways permits switching to higher data rates]{
\includegraphics[height=3.5cm]{figures/diversity_battery}
\label{fig:diversity-battery}}
\compactimg
\caption{Benchmarks}
\label{fig:results}
\compactimg
\end{figure*}

\subsection{Diversity Gain}
\label{sec:diversity-gain-eval}

Next, we evaluate the  gain introduced by Charm in SNR when coherently
combining the multiple transmissions from geographically diverse receivers at
a central server. These benchmarks are completed on a testbed covering 0.6
sq.km. using an ensemble of 8 user-deployed gateways equipped with our custom
LPRAN hardware. This testbed spans multiple buildings and open spaces between
them, and is supposed to emulate a dense urban deployment. We measure the mean
and standard deviation in SNR improvement as a function of number of gateways
across experiments.

Our results reveal remarkable SNR improvements, which as expected improves
steadily with an increasing number of gateways as depicted in Figure
\ref{fig:diversity-gain}, for clients in different locations across multiple
spreading factors. We note that the SNR gain improvement is logarithmic given
that it is measured in dB (a log-scale). Across experiments, Charm gave an
average observable improvement of 1 dB with the addition of each new receiver.
This improvement is valuable, given that every 3 dB of gain allows us to use
the next spreading factor. Any increase in spreading factor halves the
transmission air time and the resulting energy expenditure.
\figref{diversity-battery} depicts the improvement in battery life of an
indoor LoRaWAN client with an increasing number of gateways collaborating to
decode its signal. We observe that the battery life for a device transmitting
5 messages per hour at SF11 improves from 2.5 years to 10 years (SF9) with 4
or more collaborating gateways


\subsection{Range Improvement for Indoor User-Deployed Gateways}

\begin{table}[t]
\centering
\begin{tabular}{r|c|c|}
\cline{2-3}
\multicolumn{1}{l|}{}            & SF7            & SF10           \\ \hline
\multicolumn{1}{|r|}{LoRaWAN}       & \textless60 m  & \textless60 m  \\ \hline
\multicolumn{1}{|r|}{Charm with 4 gateways} & \textless60 m  & \textless100 m \\ \hline
\multicolumn{1}{|r|}{Charm with 8 gateways} & \textless200 m & \textless200 m \\ \hline
\end{tabular}
\caption{Range in congested indoor urban settings}
\label{tab:range}
\compactimg
\compactimg
\end{table}

\begin{figure*}[!t]
\centering
\subfloat[Dense cells]{
\includegraphics[height=3.25cm]{figures/dense_IncreaseHeatmap}
\label{fig:dense-improvement}}
\hfill
\subfloat[Sparse cells]{
\includegraphics[height=3.25cm]{figures/sparse_IncreaseHeatmap}
\label{fig:sparse-improvement}}
\hfill
\subfloat[Random placement]{
\includegraphics[height=3.25cm]{figures/random_IncreaseHeatmap}
\label{fig:random-improvement}}
\hfill
\subfloat{
\includegraphics[height=3.25cm]{figures/heatmapLegend}}
\hfill
\setcounter{subfigure}{3}% to se counter to (d) due to skipped caption above
\subfloat[Summary of trace-driven simulations]
\begin{tabular}[b]{r|c|c|c|}
\cline{2-4}
\multicolumn{1}{l|}{} & \begin{tabular}[c]{@{}c@{}}dense\\ cells\end{tabular} & \begin{tabular}[c]{@{}c@{}}sparse\\ cells\end{tabular} & random \\ \hline
\multicolumn{1}{|r|}{\begin{tabular}[c]{@{}r@{}}increase in \\ coverage\end{tabular}} & 46.60\% & 97.85\% & 74.59\% \\ \hline
\multicolumn{1}{|r|}{\begin{tabular}[c]{@{}r@{}}data rate +1\\ (energy/2)\end{tabular}} & 35.33\% & 38.82\% & 33.70\% \\ \hline
\multicolumn{1}{|r|}{\begin{tabular}[c]{@{}r@{}}data rate +2\\ (energy/4)\end{tabular}} & 22.30\% & 0\% & 25.82\% \\ \hline
\multicolumn{1}{|r|}{\begin{tabular}[c]{@{}r@{}}data rate +3\\ (energy/8)\end{tabular}} & 2.26\% & 0\% & 3.48\% \\ \hline
\end{tabular}%
}
\label{table:charm-improvements}
}
\vspace{-10pt}
\caption{Improvement in coverage area and data rates due to Charm in three sample deployments: (a) planned dense, (b) planned sparse and (c) random arrangement of gateways.}
\label{fig:charm-improvement}
\compactimg
\end{figure*}

In typical urban settings, users would deploy large number of LoRaWAN
gateways. Indoor settings reduce the range of a LoRaWAN device and the data
rate it can support even for short distances of tens of meters. We deploy
Charm in a congested urban building and demonstrate that collaboration can
improve the maximum range the LoRa device can support at any given data rate.

In this test, we compare a group of regular LoRaWAN gateways that
independently decode transmissions against Charm coherently combining signals
from an ensemble of 4 and 8 gateways. The distances reported in each case are
between the transmitter and the closest gateway. Our results are shown in
Table~\ref{tab:range}. Note that the ranges we observe here are smaller than
outdoor gateways, owing to attenuation inside buildings and transmission power
limits on small portable battery-powered devices. In this context, a regular
LoRaWAN gateway can service client up to approximately 60~m away. In contrast,
Charm consistently supports higher maximum range at each spreading factor. The
results with Charm using eight collaborating gateways show a marked
improvement in range of 200 m, higher than 4 collaborating gateways at 100 m.


\subsection{Effect on Coverage and Device Data Rates}
\label{sec:coverage-data-rate-improvement}

In this section, we use trace-driven simulations to show the advantages of
Charm in improving coverage area and client energy consumption in both planned
and unplanned gateway deployments. The signal power at any given receiver is
estimated using the log-distance path loss model. The model is calibrated
using 4850 points collected in a varied urban environment at different data
rates and spreading factors using GPS-connected LoRa client devices. The
log-distance parameters are $L_0  = 98.0729 dB$ for $d_0 = 40.0 m$, $\gamma =
2.1495$ and flat fading $\sigma^2 = 100.0724$. Sensitivity values for the
gateway are taken from \cite{Bor2016} to determine the SNR threshold required
to decode a transmission. In an urban environment with many obstacles and
reflectors, we observe a maximum range of 3.77 km with a transmit power of 15
dBm as opposed to the marketed range of 10 km with line-of-sight. As we are
interested in the trend of changes, we provide an optimistic estimate and
ignore the effects of fading in the simulation (assume $\sigma^2 = 0$).

We perform simulation using three sample deployment scenarios.
\figref{dense-improvement} is an ideal planned deployment where gateways
are placed in a dense hexagonal grid 6.53 km apart from each other
(corresponding to $2*3.77*\cos(\pi/6)$ km). Such an arrangement, popular in
cellular deployments, provides optimal coverage with no gaps when using an
independent decoding scheme, like in LoRaWAN. \figref{sparse-improvement}
shows a planned sparse cellular arrangement with gateways 10.05 km apart from
each other, and can provide gap-free coverage with coherent combining.
\figref{random-improvement} is a randomly-generated unplanned deployment, as
would be created by user-deployed gateways.

Assuming a fixed transmit power of 15 dBm on the client device,
\figref{charm-improvement} shows the region where Charm's local detection
followed by joint decoding shows an improvement in either coverage, client
data rates or both compared to independent decoding on gateways. The dotted
regions show regions which are covered by regular LoRaWAN while the hatched
regions are covered by Charm. Imagine the regions with no coverage with
LoRaWAN having a data rate of DR``-1'' = 0 bps (the next data rate is DR0 =
960 bps using SF = 12). The colored patches are regions where Charm shows an
increase in data rates, with the darker red areas showing larger improvements
than the lighter yellow areas. As seen in each of the sub-figures, Charm shows
an improvement in the coverage area (hatched regions are larger than the
dotted regions), an increase in client data rates (colored ares present inside
the dotted regions) as well as both simultaneously (colored areas outside
LoRaWAN's dotted coverage area).

Some specific examples of Charm's improvements are as follows: In the planned
dense deployment of \figref{dense-improvement}, Charm improves coverage area
by 46\% and substantially boosts the data rate around the centroid areas. For
the planned sparse deployment of \figref{sparse-improvement}, Charm allows us
to increase the inter-gateway distance to 11.92 km and still maintain gap-free
coverage (a decrease in gateway density by a factor of $(11.92/6.53)^2=3.33$).
With an unplanned deployment such as in \figref{random-improvement}, Charm not
only improves coverage and data rates but also manages to fill in islands and
orphaned regions with coverage. This is particularly relevant to urban regions
where areas of bad coverage are formed in building basements and other indoor
regions as seen in \figref{penetration-test}. These examples provide an
insight to Charm's substantial benefits to existing and future LPWANs. A
detailed summary of these results is shown in
Table~\ref{table:charm-improvements}. Improvements are reported as percentages
with reference to the area covered by LoRaWAN in each deployment. Every
increase in the data rate, doubles the battery life of a client device. Some
regions in the simulation show up to 8 $\times$ energy savings.

% \figref{dense-improvement} shows an ideal planned deployment where gateways
% are placed in a dense hexagonal grid 6.53 km apart from each other
% (corresponding to $2*3.77*\cos(\pi/6)$ km). Such an arrangement, popular in
% cellular deployments, provides optimal coverage with no gaps when using an
% independent decoding scheme. However, the points farthest from the gateways
% (e.g. on the centroid between three neighbouring gateways) have to use the
% slowest data rates (corresponding to SF12) and their battery life
% correspondingly suffers. If such a deployment were to be augmented with Charm,
% the region between the gateways can all communicate using SF9 or better. For
% the centroid points, this reduces the transmit time by approximately a factor
% of 8 and leads to huge energy savings. Additionally, we see the coverage area
% for joint decoding has expanded beyond the original coverage region. Some of
% the devices in the expanded coverage area can even communicate with higher
% data rates.
% An interesting phenomenon seen in each of the sub-figures are the
% thin concentric circles of improvement around each gateway. These are regions
% just outside the original boundaries covered by any given spreading factor
% that also gain a data rate increase due to joint decoding. However, this
% effect is not as prominent and the circles are small.

% \figref{sparse-improvement} shows a sparse cellular arrangement with gateways
% 10.05 km apart from each other that can provide gap-free coverage. Charm thus
% enables decreasing the gateway density by a factor of 3.33 (proportional to
% square of inter-gateway distance $=(11.92/6.53)^2$) while providing the same
% level of coverage. Finally, we show in \figref{random-improvement} how Charm
% also improves the performance of an unplanned deployment of user-deployed
% gateways by improving coverage area as well as data rates for clients. Another
% interesting phenomenon to observe here is that Charm's joint decoding manages
% to fill in islands and orphaned regions. This is particularly relevant to
% urban regions where areas of bad coverage are formed in building basements and
% other indoor regions as seen in \figref{penetration-test}.
\section{Conclusion and Future Work}
\label{sec:conclusion}

This paper presents Charm, a novel system that improves battery life and range
in LP-WANs clients. Charm achieves this through a mechanism that pools
together weak received signals across multiple gateways at the cloud in order
to jointly decode them. In doing so, Charm overcomes multiple challenges,
including a hardware-software design to detect weak signals at the gateway,
and ensure scalability at the cloud. A pilot evaluation of Charm on a network
of twelve LoRaWAN gateways serving a large neighborhood of a major U.S. city
demonstrates a large improvement in coverage and client battery-life.

An interesting side-benefit of Charm is its impact on scalability of the
network overall. Given that Charm improves coverage, one might expect a large
number of collisions from transmitters who newly gain coverage with existing
ones. Counter-intuitively, this is not the case because Charm allows devices
across the board to transmit at faster data rates, increasing available air
time in the network. Our future work will explore further improving network
scalability along two dimensions: (1) First, a full-scale distributed MIMO
system atop LP-WAN in the cloud, that can also handle collisions from a large
number of clients. (2) Second, offloading of TV whitespace spectrum at peak
demand, based on an FCC license recently granted to our university.



% This paper presents a novel location-aware network management system for
% LoRa-class LP-WANs operating in unlicensed and whitespace frequencies. It
% presents an RF-based localization system that stitches together information
% across frequencies to improve positioning accuracy of LoRa clients, even
% without access to their channel state information. We build on the
% localization framework to build a resource allocation system for LP-WAN that
% efficiently allocates wireless resources subject to FCC's regulatory
% constraints.  Our network-management system  is designed to be
% location-aware, exploiting live measurements to identify and respond to
% interference and help network managers plan changes to deployment. Our
% system was implemented and deployed in a large university campus and results
% from proof-of-concept experiments and large-scale trace-driven simulations
% are presented. In the future, we aim to expand our deployment to incorporate
% transmissions on the white spaces and are currently engaged in efforts to
% procure the relevant licenses. We also intend to integrate smart sensor
% devices currently deployed on-campus with LoRaWAN radios to evaluate our
% network management platform live and at-scale.

\section*{Acknowledgment}

This research is supported in part by the National Science Foundation under
award CNS-1329644 and the CONIX Research Center, one of six centers in JUMP, a
Semiconductor Research Corporation (SRC) program sponsored by DARPA.

\bibliographystyle{ACM-Reference-Format}
\bibliography{references} 

\end{document}
