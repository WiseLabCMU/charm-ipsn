\section{OLD STUFF: Motivation}
\label{sec:old-net-scale}

{\color{blue} A quick introduction to LoRa.... Check if we need more details about a particular aspect.}

LoRa offers a unique collection of features for dynamic data-rate selection and concurrent reception of packets on the gateway. In the U.S., LoRaWAN standardized 64 channels of downlink (125$khz$) and 8 channels of uplink (500$Khz$) per base station. Current gateways use the Semtech SX1301 base band processor which can listen on up to 8 channels at the same time. Most gateway hardware pairs the SX1301 with two radio frontends that each have four programmable reception channels. Each radio front-end listens across 800$kHz$ of bandwidth which supports up to four independent 125$kHz$ channels or a single 500$kHz$ channel (usually used for uplink).  The two radios can be configured independently such that one operates on the 433$MHz$ band while the other at 900$MHz$ which allows for both a large bandwidth span as well as whitespace coverage.  LoRaWAN specifies 7 spreading factors per channel (SX1301 supports 6) that in a similar spirit to CDMA, allows for multiple packets at different rates to be received simultaneously on the same channel.  Each increase in spreading factor leads to a doubling of transmission time but a 3.5$dB$ increase in link budget (i.e. range).  FCC regulations allow LoRa radios to operate in {\em hybrid-mode} where they can hop across a reduced subset of the total LoRaWAN channel set (8 sequential channels instead of 64). We refer the reader to \cite{sx1301} for a more detailed description of LoRa radio hardware capabilities.

{\color{blue} Some whitespace discussion would follow...}

% These parameters, along with FCC regulations for LoRa provide several degrees of freedom that can be used to optimize network performance in four main ways.  First, Adaptive Data Rate (ADR) algorithms can be used to improve network capacity by configuring power levels and selecting spreading factors.  Second, most standard base stations operating in hybrid-mode which can be coordinated across a network to avoid interference and collisions.  Third, clients can coordinate with the infrastructure to more intelligently select a base station to associate with, based on estimated load and spreading factors. In cases of high client density, we see that reducing data rate by shifting clients to slower spreading factors can lead to improved overall capacity. Finally, powered clients can help localize uncoordinated base stations or noise sources that the network can avoid.

Since LoRa radios can span a wide frequency range, we can also access available whitespace depending on the region and usage schedules.  Whitespace access requires location information for registering with spectrum databases and requires an out-of-band mechanism (like LoRa) as part of the join processes.  This makes it an ideal mechanism for offloading traffic in support of scale.

% In the rest of this section, we evaluate the potential to exploit the above techniques to improve overall network capacity. We first experimentally profile the LoRa channel in terms of interference between channels and spreading factors.  Next, we look at how base stations can be more efficiently managed and how clients can more efficiently be allocated to them.  We then show an initial mechanism that leverages our localization approach to perform impact assessment for noise sources.

% [ANTHONY, ADWAIT, MAX, ARTUR]

% \subsection{An Additive Interference Model}
% \label{sec:AIM}

% {\color{red} These effects don't seem to matter much once we consider probabilities.}

% A fundamental property of wireless communication is the use of a shared medium, in which the reception of a signal can be affected by the transmission of other unrelated signals. A wireless receiver decodes an incoming signal by treating all other interfering signals as noise. A large amount of interference can drastically lower the chance of a signal being successfully received, thus the interference characteristics of a network often have a significant impact on its overall performance. This also means that the choice of interference model is important when simulating a wireless network, as it can directly affect the accuracy of simulation results.

% The {\it Independent Interference Model (IIM)}, also called the {\it protocol model} of interference, is a widely-used model which computes the level of interference by performing multiple independent pair-wise comparisons between the power of the incoming signal and the power of each interfering signal. While this method can be computationally expedient, it does not accurately describe how real signals interact in a wireless medium. Interference from wireless signals is typically cumulative, and multiple interfering signals "add up" to a greater level of interference. This phenomenon is not accounted for by IIM, making it an inappropriate to evaluate large wireless networks, especially in terms of throughput or capacity.

% To address the limitations of IIM, we employ the {\it Additive Interference Model (AIM)}, also called the {\it physical model} of interference. In AIM, when two or more interfering signals are present, their power is aggregated and the result is compared with the power of the desired incoming signal. This more closely mimics the interaction between multiple signals at a physical wireless receiver. Whereas IIM adjusts signal reception probability based on only the worst interferer (in terms of its power or proximity), AIM takes the more accurate approach of incorporating all interference. This distinction is less consequential for modeling low-density networks in which transmissions are unlikely to overlap, but becomes much more so for denser networks. By incorporating all interference, AIM also more accurately models the capture effect in the presence of many concurrent transmissions. LoRa networks are characterized by both high densities of devices and a reliance on the capture effect, and thus should be simulated using AIM. More discussion of AIM and its efficacy relative to various forms of IIM can be found in~\cite{iyer2006}.

%We conducted our experiments using LoRaBUG nodes (shown in \figref{lorabug-photo}) and a LinkLabs LoRa SX1310 gateway.

%Additionally, careful power control is required in order to minimized the amount of interference between Spreading Factors.

%Describe the additive Interference Model (Multiplication Matrix), that is later use for simulation purposes
%}


%{\color{blue}

%\begin{itemize}
%\item Why do we need the model? To accurately capture network capacity at high density of devices. LoRaWAN is actually capable of reaching these densities, and thus we need to account for all possible interference.
%\item How does it differ from existing models? Regular models consider interfering sources as being independent. This does not hold for high densities when more than 2 transmissions may interfere with each other.
%\item How the interpretation of the capture effect changes in these scenarios? The power of interfering transmissions may add up so that the capture effect may no longer dominate over weaker transmissions.
%\end{itemize}

%The model makes some assumptions:

%\begin{itemize}
%\item Interference is uncorrelated and thus their power is linearly additive
%\item Messages are dropped if the combined interference overpowers the message at any point of time
%\end{itemize}
%}

\subsection{Base-Station Management}\label{sec:bsm}

{\color{blue} Section is relevant but content needs to be updated}

Most current LoRa base-stations only support traffic on 8 out of the 64 total LoRaWAN channels simultaneously. Each base-station has the freedom to select which set of 8-channels to operate on.  Ideally, neighboring base-stations should operate on different channel sets if any of their coverage regions overlap.  In practice, this can be difficult to estimate without collecting propagation information in the field or without adopting pessimistic unit-disc models.   

One approach for assigning frequencies is to solve the graph coloring problem where each base-station is a node and a link is added if its region of coverage overlaps with any other base-stations nearby.  Each of the eight sets of frequencies represents a color.  Unfortunately, solving the graph coloring problem is NP-complete, but multiple heuristics perform quite well in practice.  Our approach selects the node with the highest degree in the graph (or an arbitrary high-degree node in case of a tie) and then perform a breadth first search (BFS) traversal across the graph where at each node a color is selected using a least recently used (LRU) policy that does not conflict with any neighboring nodes.  If there are no available colors, the least recently used color is placed in conflict. This would occur in extremely dense networks like apartment complexes that could have hundreds of set-top boxes with LoRa gateways.  This heuristic tends to spread the colors apart across the network. 

Base-stations can periodically scan the RF environment around them.  If they detect significant interference on their current channel, they could potentially change to a different frequency set.  Unfortunately, changing channels at runtime imposes an energy penalty on clients that may sporadically attempt to connect and find that the network is unreachable and be forced to rejoin.  This process can be optimized if several 64-channel gateways are deployed across a region to help manage this join process.  If all gateways use 64-channel hardware then graph coloring would no longer be required, but fine-grained power control would become increasingly critical.




% \subsection{Client to Base-Station Association}
% \label{sec:bs-assoc}

% One of the most important factors to consider in a multiple base-station LoRa network is deciding how client devices should associate with base-stations during the join process.  The current method used by most LoRaWAN servers adopts the common practice in WiFi and Cellular networks where a device greedily joins the base-station with the highest signal strength.
% In the case of a high load on one base-station and low load on its neighbor the greedy association scheme leads to a decrease in overall performance.
% Our network management scheme models clients joining base-stations as a bin-packing problem where each client is allocated an amount of capacity on a channel and particular spreading factor.  A bin-packing algorithm is then responsible for both filling up channels as well as spreading factors on each gateway as clients try to join the network.

% There is no known polynomial solution to the bin-packing problem, but there are many heuristics.  We evaluate the First-Fit heuristic against the commonly used greedy selection. The different spreading factors used in LoRa provide a level of orthogonality and we can model them as independent bins in each frequency band. However, the airtime of packets increases by a factor of 2 for each spreading factor and the bins are scaled proportionally to consider this. We order buckets in the same spreading factor, together in a group, before considering the next spreading factor for the First-Fit heuristic. In an environment with a high density of nodes, this approach can associate nodes to a farther base-station at the cost of data rate


% \subsection{Interference Mitigation} \label{sec:interferemit}
% Unlike cellular networks, LP-WANs operate in the unlicensed bands, making them susceptible to cross-technology interferers: RFIDs, 802.15.4 radios, as well as neighboring or rogue LP-WAN interferers. Further, deploying LP-WANs in the whitespaces makes detecting interference crucial, since regulation requires users to abort transmissions on frequencies where interference is detected ~\cite{FCC_Whitespaces}. 

% Our approach to mitigate interference relies on exploiting the scale of the network and the properties of LoRaWAN to our advantage. On the downlink, we capture signal measurements from wall-powered client nodes that deliver signal-to-interference plus noise (SINR) measurements periodically to the base station on different frequencies. This data, coupled with the locations of the users is used to build a spatial heatmap of interference in different locations. We show how such a heatmap can be used to assess the number of nodes impacted by an interferer, even if several of these nodes remain idle for an extended period of time. On the uplink, we constantly monitor and identify interfering sources using software radios at the base stations. The availability of TV whitespaces expands the range of frequency options on the downlink and uplink, increasing the available strategies for interference avoidance.\\\vspace*{-0.1in}  

% \noindent \textbf{Interference Impact Assessment: } Our approach to detect interfering sources, crowd-sources the task to wall-powered clients that are already monitoring spectrum on the downlink frequencies. The base station aggregates this information along with the location of the said clients. It can then estimate geometric bounds of the interference and compute the number of clients impacted by this interferer on the downlink. Given that the locations of the clients are known a priori using our localization framework, such a mechanism can detect interference perceived even by clients that have been asleep since the interferer appeared. 

% Implementing the above design however, is challenging if wall-powered clients are not uniformly available through the network. In this case, one cannot accurately estimate statistics on the interference impact given that we have very few samples of the interferer's signal. Our solution exploits the algorithm described in Sec.~\ref{sec:localization-inter} to localize interfering sources using as little as 3-4 wall-powered clients. One can then use the power of interference perceived by these clients relative to the base station to gauge the approximate sphere of influence (using the Euclidean disk model~\cite{chen2011implications}) of the interfering source, centered around its location. \\\vspace*{-0.1in}  

% \noindent \textbf{Interference Avoidance Strategies: } Once the source of interference is detected, interference avoidance algorithms are triggered at the base station. On the uplink, software radios detect the source of interference on a given set of frequencies. However, complicating matters is the presence of multiple frequencies, each with different interference-to-noise values. Our approach solves this optimization problem by extending the graph coloring problem in Sec.~\ref{sec:bsm} above. We replicate nodes to account for the different sets of non-interfering frequencies. We perform a similar optimization on the downlink as well, while accounting for the availability of white space frequencies, unlike the traditional LoRaWAN specification. 

% \subsubsection{Downlink interference}

% \begin{itemize}
% \item How do we identify down-link interference? Use wall-powered nodes.
% \item How do we deal with down-link interference? Break out of spec and use other frequencies. Note As shown in a later figure, we see that the existing node radios are able to communicate on some white space frequencies as well.
% \end{itemize}

% \subsubsection{Interference on the same Frequency}

% \begin{itemize}
% \item Concurrent transmissions on the same frequency is possible by using different Spreading Factors, but additional interference is generated.
% \item The amount of Signal-to-Interference-plus-Noise-Ratio (SNIR) should be considered to determine if a transmission is successfully received.
% \item Capture effect is observed on the same frequency and same Spreading factor if the received power of one transmission is more than 6dB higher than the other.
% \end{itemize}

% \subsection{Network Planning}
% Our approach also leverages the scale of the network to facilitate planning base station deployment, without resorting to drive tests. Consider the problem of assessing the impact of placing a base station at a given location, based on say, the locations of clients with poor performance in the network. One would then like to estimate how many clients can expect improved performance owing to this base station. Yet, evaluating this would require placing a physical base station at this location, since signals from certain transmitting sources at this location could be occluded. 

% In contrast, our approach uses a team of wall-powered clients at a physical location to estimate how many clients a base station in their location can listen to on the uplink. We use our localization framework to find the position of the powered clients in the vicinity of the planned location of the base station. We tune these clients to the uplink frequencies to estimate the set of clients that are in range. We then estimate the number of clients a base station at this location would serve, while taking into account the difference in range of the clients vis-a-vis a base station. A key challenge in implementing this approach in the LP-WAN context, is that different client radios may employ different transmit powers and spreading factors, based on their preset geographical locations and the base stations they are currently associated with. Our approach therefore co-opts these parameters into the estimation process, accounting for increased range of clients that are received at artificially lower power, owing to their current transmit settings (and vice-versa). 

