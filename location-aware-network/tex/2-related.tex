\section{Related Work}
\label{sec:related-work}

% {\color{blue}
% [ANYTHONY, SWARUN, BOB, ARTUR, ADWAIT, DIANA, MAX]

% Particular topics to consider
% \begin{itemize}
% \item PerCom workshop paper
% \item Localization (Reverse GPS)
% \item WiFi and Cellular existing work
% \end{itemize}
% }


\noindent \textbf{Low-Power Wide-Area Networks: } Recent years have seen much interest in Low-Power wide area networks (LP-WANs), including the development of new hardware and standards. Private enterprises such as Semtech~\cite{Sornin2015, LoRaWanAlliance2015} and SigFox~\cite{sanchez2016state} have developed LP-WAN chips that use extremely narrow bands of unlicensed spectrum. In contrast, cellular standardization bodies have developed two standards for LP-WAN communication  for cellular base stations to communicate with low-power IoT devices over licensed spectrum:  LTE-M~\cite{GSMAssociation2016} and NB-IOT~\cite{Ratasuk2016}. Unlike LoRa and SigFox, these technologies require devices to periodically wake up to synchronize with the network -- a burden on battery life.

Several recent measurement studies have been conducted to evaluate the performance and range of LP-WAN networks~\cite{petric2016measurements, 7499263, toldov2016performance} and perform theoretical capacity analysis~\cite{mikhaylov2016analysis}. Early pilot deployment efforts are also underway with
SigFox deploying their hardware to connect security alarms to the cloud in Spain~\cite{sanchez2016state}, smart blood refrigerators in the Democratic Republic of the Congo~\cite{ramachandranmupnp} and smart city applications \cite{centenaro2015long}. These efforts motivate the challenge of scalability of LP-WANs, particularly in dense urban scenarios. While recent trends such as the opening up of the TV whitespaces~\cite{FCC_Whitespaces} has resulted in increased spectrum, there remains the need to study the scaling limits and spectrum management strategies for city-scale LP-WANs. Our paper addresses this lacuna in literature by exploring LP-WAN spectrum and network management over unlicensed spectrum including the TV whitespaces. \\\vspace*{-0.1in}

\noindent \textbf{Localization in LP-WAN: } The problem of RF-localization has received much attention over the years. The variety of systems have been proposed using Wi-Fi~\cite{kumar2014accurate,vasisht2016decimeter, xiong2013arraytrack}, cellular~\cite{kumar2014lte,sun2005signal}, and Bluetooth~\cite{Kalliola2006, lazik2015alps, bargh2008indoor} to name a few. Outdoors, localization has been dominated by GPS that provides meter-level accuracy. However, such systems are not designed to support long-range low-power communication powered by small, ten-year batteries. Past systems for sensor positioning~\cite{sichitiu2004localization, gholami2013rss} and wildlife tracking~\cite{andrade2011reverse, maccurdy2009automatic} operate over long distances at low-power outdoors, yet lose accuracy in urban and indoor environments. Indeed, there remains a gap for  accurate RF-based positioning  that spans both indoor and outdoor spaces that is suited to low-power wide-area networks. 

The proposed work aims to bridge this gap by presenting a system that is simultaneously low-power, long-range and accurate (to few meters) across indoors and outdoors. It does this using radios that span just few hundred kHz of bandwidth, and therefore remain low power. Our work builds on past solutions that combine information across frequency bands in the WiFi context~\cite{vasisht2016decimeter}, but demonstrates performance with significantly lower power budget and localizes radios that do not measure wireless channels themselves. Further, it leverages past work on network localization~\cite{nwlocalization,sethnwlocalization} to position user devices using their mutually recorded times-of-flight, even if some of them are not in the range of multiple base stations.  \\\vspace*{-0.1in}

\noindent \textbf{Network Management in WiFi and Cellular: } Much of past work has considered the problems of spectrum allocation and frequency tiling in cellular~\cite{mcdiarmid2000channel, khanna1998wireless} and WiFi networks~\cite{baid2015understanding, mishra2005weighted}. With the opening up of the TV whitespaces, efforts to build cognitive radios and deploy collaborative spectrum sensing have gained traction~\cite{yucek2009survey, nekovee2010cognitive}. Recent systems have also considered crowd-sourcing to  monitor cellular networks ~\cite{fatemieh2010secure, shi2014crowdsourcing}. 

Our work complements past literature by tailoring spectrum management to the distinctive limitations of LP-WANs. We explicitly account for frequency hopping and spread spectrum techniques, unique to LoRa LP-WAN. We also exploit the scale of the network to perform collaborative sensing, interference detection and network planning, keeping in mind the power constraints of LP-WAN. 

