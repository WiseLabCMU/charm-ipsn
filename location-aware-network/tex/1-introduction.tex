\section{Introduction}
\label{sec:intro}

\begin{comment}
{\color{blue}
[ANTHONY, SWARUN, BOB]

Key points to cover
\begin{itemize}
\item LPWAN is emerging
\item Unlicensed spectrum $\rightarrow$ uncoordinated
\item Uncoordinated communication with long range $\rightarrow$ poor scaling
\item Coexistence is critical
\item LPWAN is device centric and energy constrained
\item LPWAN is like WiFi (free for all) but in the cellular arena (city-wide/ubiquitous)
\item Contributions are:
\begin{enumerate}
\item Benefits of coordination
\item Network management
\item Techniques and limits of LoRa localization
\item New platforms
\end{enumerate}
\end{itemize}
}
\end{comment}

Low Power Wide Area Networks (LP-WANs) are increasingly seen as an attractive communication platform for city-scale sensor deployments. LP-WANs offer the ability to wirelessly connect energy-poor sensor devices to base stations over distances of many miles. In deploy-once, never-touch-again, low bandwidth sensing applications, LP-WANs offer power and cost advantages over cellular networks.  While deployments are far from pervasive, the capabilities of LoRaWAN~\cite{Sornin2015, LoRaWanAlliance2015}, SigFox~\cite{centenaro2016} and Ingenu's RPMA~\cite{Ingenu2015} have attracted investment and have spawned early deployments.  

Across the board, LP-WAN technologies have adopted two key design principles at their core: a device-centric architecture and the use of unlicensed spectrum. First, LP-WANs are device-centric, giving the device the freedom to sleep (and save power) as it sees fit. This provides them an edge over competing 3GPP approaches such as LTE-M~\cite{GSMAssociation2016} and NB-IoT~\cite{Ratasuk2016} that are network-centric, requiring the devices to spend energy checking in regularly, even when they have no traffic to pass. Second, LP-WANs take advantage of  unlicensed, shared spectrum.  LoRa uses sub-GHz spectrum -- 433 and 868 MHz in Europe, 430 MHz in Asia, and the 915 MHz ISM band in the U.S.A. -- known for excellent long-range propagation characteristics (in contrast to WiFi at 2.4 GHz).\footnote{Similarly, SigFox -- a European effort -- uses the 868 MHz unlicensed band in Europe and the 915 MHz ISM band in the U.S.A.} Combine this with  the opening of TV-whitespace frequencies around the globe~\cite{FCC_Whitespaces}, and LP-WANs may bring us closer to the vision of an easy-to-deploy city-wide network enabled by inexpensive spectrum, serving smart sensors that are simple, low-cost, and low-power. 

%Importantly, because it is a spread-spectrum technology, LoRa can take advantage of higher transmit power levels subject to an FCC-imposed constraint of frequency hopping (47 CFR 15.247).  


%Combine this with  the opening of TV-whitespace frequencies~\cite{FCC_Whitespaces} and the frequency agility of LP-WAN chipsets, and one realizes that spectrum availability may not be the primary limiting factor in LP-WAN deployments.  In the USA, LoRa operates in the 915 MHz ISM band (902-928 MHz). 

Yet, achieving this vision remains challenging owing to two key design constraints -- the need for scalability and compliance to regulation. First, LP-WANs, with their low cost, good coverage and easy deployment, are likely to scale up rapidly, with the number of connected devices out-stripping cellular networks. However, operating at such scale in a device-centric network where nodes wake up and transmit on their own schedules can lead to collisions and channel saturation \cite{Abramson1970}. Second, the use of unlicensed spectrum is subject to important regulatory constraints. In particular, to transmit at high power on the sub-GHz unlicensed bands, the FCC-imposes a frequency hopping requirement\footnote{FCC Part 15 Regulations: 47 CFR 15.247}, complicating spectrum management. Further, transmitting in white spaces requires a node to be aware of its location to avoid interfering with licensed primary transmitters. Designing a system that manages spectrum access in LP-WANs without sacrificing its power advantages from device-centricity remains a fundamental challenge. 


%Given this, will the presumed power advantages of device-centric LP-WANs simply vanish?  How can we share channels effectively while preserving the power advantages of device-centricity?  We must find effective ways to coordinate spectrum access and to coexist.

%In this work, we examine anticipated scale and the frequency-hopping mandate as unique characteristics of LP-WANs that can be utilized to develop localization tools and to create channel management mechanisms.

This paper presents a location-aware network management systems for LP-WANs that embraces the dual constraints of scale and regulation to its advantage. We first present an RF-based localization system that exploits the frequency hopping requirements of wireless devices to stitch together information across frequencies and thereby improves positioning resolution. We demonstrate how our approach achieves high accuracy, both indoors and outdoors, without imposing the battery drain or cost of GPS chipsets.  This is especially valuable for applications in which sensor devices may be installed in bulk, say in a smart city.  To be able to correlate the information with the environment, each device's position must be available, even for devices that do not move over their lifetime--for instance, traffic flow sensors.  Imposing the requirement to geo-reference each device at installation time may represent a significant cumulative labor cost.  Using a GPS in each sensor, even if it is only used once, adds to device cost and size.  If we can make localization a network function and demonstrate suitable accuracy, these costs can be mitigated or eliminated altogether.  

Next, we leverage the  scale of an LP-WAN network with a large number of devices at known locations to obtain a live spectrum map over a wide-area. We show how this facilitates a range of network management decisions -- spectrum allocation, network planning and interference mitigation, without imposing additional device complexity or relying on out-of-band mechanisms such as drive tests. We implement our system on the commodity LoRa LP-WAN system, given its spread-spectrum nature and compliance to FCC hopping norms. We present both experimental results to demonstrate system feasibility as well as trace-driven simulations to validate performance at scale. In order to promote community adoption,  our system is entirely LoRaWAN compliant. 

\emph{Localization}: At the heart of our network management approach is a solution that accurately locates sensing devices indoors and outdoors, without the need for power-hungry GPS chipsets. Our solution embraces the scale as well as frequency hopping constraints imposed on LP-WANs. Specifically, it is well known that the precision with which one can localize a wireless device is a function of the signal bandwidth \cite{Krizman1997}. However, LP-WANs are designed to be narrowband (hundreds of kHz at best~\cite{alliance2015lorawan}), owing to stringent power requirements on hardware, limiting localization resolution to hundreds of meters. In contrast, our approach embraces mandatory hopping  to gather 
and combine signal information across frequencies to improve precision in localization. In particular, our mechanism uses this information to estimate the time-of-flight of signals between pairs of LP-WAN enabled sensor devices -- which provides their relative distances. It then combines pair-wise distances observed across the network, including to base stations and devices at fixed and known locations to precisely locate the remaining nodes. In doing so, our approach overcomes multiple challenges, such as hardware non-idealities and the fact that some LP-WAN radios are not designed to provide low-level signal information. Further, our approach exploits the scale of the network to ensure that clients can be localized even if they are not in communication-range of multiple far-away LP-WAN base stations, saving the energy drain such a requirement would impose. 


%We develop a mechanism to compute time-of-flight accurately between pairs of LP-WAN radios, which can then be used to compute their relative distances accurately. we develop a solution to accurately estimate time-of-flight between pairs of LP-WAN radios using RF-signal measurements gathered across multiple frequencies. Our approach demonstrates high accuracy in doing so, given that it. we develop a solution that exploits the frequency hopping of pairs of LP-WAN radios to. 

%The precision with which one can localize a wireless device is a function of the signal bandwidth \cite{rappaport}.  Mandatory hopping can be used to gather  and combine signal information across frequencies, resulting in more precise localization.  Further, with scale, it becomes possible to combine range information across nearby clients to infer location.

\emph{Management}: Our approach to network management directly benefits from the scale of LP-WANs and regulation-mandated frequency hopping. By systematically collecting channel state information across devices, real-time large-scale channel state maps can be created, tagged with locations of devices.  While hopping provides sampling across a wide range of frequencies, the scale of LP-WANs provides dense sampling spatially. 

Our system exploits the live spectrum heatmap to provide three services: spectrum allocation, network planning and interference mitigation. First, we devise an algorithm to allocate chunks of frequencies to neighboring base stations to avoid mutual interference using a graph-coloring approach. Unlike frequency tiling in the cellular space~\cite{mcdiarmid2000channel, khanna1998wireless}, our approach actively accounts for LP-WAN specific peculiarities, such as frequency hopping, spread spectrum, and interactions between multiple interfering sources. For instance, we demonstrate how cellular and Wi-Fi truisms, such as associating with the closest base station or picking the highest possible data rate for a given client, may not always be in its best interest in the LP-WAN context. Second, we facilitate better network planning by identifying optimal locations to deploy new base stations. We do so by emulating deployment of a new base station by collating signal measurements from teams of co-located LP-WAN client devices, already present in the network. This assists network managers in assessing the impact of placing a new base station in their vicinity, without resorting to expensive drive tests. Finally, we present a solution to mitigate interfering sources, by quickly identifying their presence -- using crowd-sourced measurements from sensor devices. Our network management system is intrinsically designed to only exploit sensor clients that are currently awake, preserving the power advantages of the device-centric LP-WAN architecture. 

We evaluate our system both through proof-of-concept experiments and large-scale trace-driven simulations. We implement our system on commodity LoRa base stations and clients, RTL-SDR radios \cite{rtlsdr} to obtain wireless channel measurements. We perform experiments both in indoor and outdoor spaces across a large university campus. Our results reveal the following:
\begin{itemize}
\item \textbf{Localization: } Over pairs of nodes separated by up to 40~meters across indoor and outdoor spaces, we perceive a median positioning error of 2.81~meters. 
\item \textbf{Management: } We see that  selective base station frequency allocation can improve capacity by around 5\%, load balancing schemes that utilize spreading factors can increase capacity by 5-20\%, and a single white space band can improve a standard base station's capacity by 3X.
\item \textbf{Synergy: } We see that if 10\% of the nodes in a network are powered and hence always-on, we can  predict the set of clients impacted by noise with an 80\% accuracy.
\end{itemize}

\textbf{Contributions}: This paper presents:
\begin{tightitem}
\item A novel approach to  improving network based localization for LoRa-class networks using channel stitching.  Our approach is robust against key hardware non-idealities and can localize receive-limited devices such as interferers.
\item A resource allocation system for LoRaWAN that efficiently allocates wireless resources subject to constraints from regulation (FCC hopping) and LoRa (spread spectrum).  This includes initial estimates of the potential network capacity possible with multiple coordinating LoRa base-stations that also have access to whitespace frequencies.
\item A network-management system that is location-aware, exploiting live measurements from sensor clients to react to interference and pro-actively plan network changes.     
\end{tightitem}
