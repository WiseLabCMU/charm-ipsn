\section{Gateway}
\label{sec:gateway}

{\color{blue} Points to cover
\begin{itemize}
    \item Hardware + block diagram + capability + how to decode IQ streams
    \item Local detection algorithm
    \item Optional: (LoRa-aware) compression
\end{itemize}

}

\subsection{Hardware}
\label{sec:hardware}

Existing LoRaWAN gateway hardware cannot provide the raw I/Q streams necessary
for joint decoding. Our custom platform, named \textit{LPRAN} and shown in
\figref{lpran-hardware} and \figref{lpran-block} {\color{blue} (ADD PHOTO and
BLOCK DIAGRAM)} is an auxiliary peripheral to a gateway and can provide the
necessary quadrature streams. It combines a Semtech SX1257 868/900 MHz RF
front-end {\color{blue} (add citation to datasheet)} with a Microsemi IGLOO
AGL250 FPGA for local processing. The processed data streams are then
transferred to a Raspberry Pi 2 over an SPI bus, which can then perform
additional local processing, time-stamping and temporary local storage until a
stream is requested by the joint-decoder.

The SX1257 RF front-end provides 1-bit delta-sigma modulated signals ($s_j$)
at 36 MSps for the I and Q streams. To convert these into analyzable samples
($x_i$), we sum consecutive samples in windows of size 64 and convert it into
a single 8-bit sample.

{\color{blue} Describe capabilities of the platform. How much bandwidth supported, size of streams, how is large is the output and how do we handle it. Also describe what else it can be used for.}

\begin{align*}
x_i &= \sum_{j=64*i}^{64*i + 63} s_j
\end{align*}

A window size of 64 is selected since we are only interested in a final
bandwidth of approximately 500 kHz that the RF front-end is capable of
capturing. Applying this technique, the uncompressed I/Q streams generate data
at a rate of 1.125 MBps.

\subsection{Local Detection}
\label{sec:local-detection}

WHy is it necessary?

Why is it hard?

How do we implement it?

How well does it perform?