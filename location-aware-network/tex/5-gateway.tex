\section{The \name\ Gateway}
\label{sec:gateway}

{\color{blue} Points to cover
\begin{itemize}
    \item Hardware + block diagram + capability + how to decode IQ streams
    \item Local detection algorithm
    \item Optional: (LoRa-aware) compression
\end{itemize}

}

We first describe \name's design at the gateway to ensure accurate decoding of weak clients, by relaying suspected weak signals to the cloud. \name\ achieves this first through a software algorithm at the gateway that identifies weak transmissions that may be significantly below the noise floor. We further implement this approach through in hardware by building a custom programmable platform that integrates LoRaWAN radio to a low-power FPGA at the gateway. We show how a \name-gateway can detect weak signals in real-time through this design, while simultaneously being programmable and responding to policy changes from the cloud. 

\subsection{Locally Detecting Weak Signals}
\label{sec:local-detection}
To reap the benefits of coherent diversity combining across multiple gateways, \name\ must relay weak signals to the cloud. Unfortunately, many -- perhaps all of these signals may be well below the noise floor at the gateway by as much as 30~dB, in which case we lose the benefits of coherent combining. Yet, transmitting all received signals to the cloud to overcome this problem is unfeasible given that gateways have limited uplink bandwidth to the cloud. To put this in perspective, streaming all received I/Q samples to the cloud requires an uplink bandwidth of 18 Mbps to the cloud. However, the vast majority of LP-WAN gateways are likely to be user-deployed hardware such as set-top boxes that cannot afford this bandwidth. Indeed, this creates  trade-off between the 

Our approach to detect weak signals relies on the structure of the LoRaWAN protocol. Specifically, LoRaWAN transmits signals through a long preamble sequence that contains chirps -- signals whose frequency increases linearly with time. 

WHy is it necessary?

Why is it hard?

How do we implement it?

How well does it perform?

\subsection{Programmable Hardware Design}
\label{sec:hardware}

Existing LoRaWAN gateway hardware cannot provide the raw I/Q streams necessary
for joint decoding. Our custom platform, named \textit{LPRAN} and shown in
\figref{lpran-hardware} and \figref{lpran-block} {\color{blue} (ADD PHOTO and
BLOCK DIAGRAM)} is an auxiliary peripheral to a gateway and can provide the
necessary quadrature streams. It combines a Semtech SX1257 868/900 MHz RF
front-end {\color{blue} (add citation to datasheet)} with a Microsemi IGLOO
AGL250 FPGA for local processing. The processed data streams are then
transferred to a Raspberry Pi 2 over an SPI bus, which can then perform
additional local processing, time-stamping and temporary local storage until a
stream is requested by the joint-decoder.

The SX1257 RF front-end provides 1-bit delta-sigma modulated signals ($s_j$)
at 36 MSps for the I and Q streams. To convert these into analyzable samples
($x_i$), we sum consecutive samples in windows of size 64 and convert it into
a single 8-bit sample.

{\color{blue} Describe capabilities of the platform. How much bandwidth supported, size of streams, how is large is the output and how do we handle it. Also describe what else it can be used for.}

\begin{align*}
x_i &= \sum_{j=64*i}^{64*i + 63} s_j
\end{align*}

A window size of 64 is selected since we are only interested in a final
bandwidth of approximately 500 kHz that the RF front-end is capable of
capturing. Applying this technique, the uncompressed I/Q streams generate data
at a rate of 1.125 MBps.

